
NPI non parametric inference

Use of Ranks and Randomisations
Wilcoxon Rank Sum  / Mannwhiteny U test
Wilcoxon signed ranl
kolmorogorov smirnoov (one and two sample)
Goodness of fit and rank correlation tests
non-parametric confidence intervals
Applications using sign and wilcoxon rank sums (MWU ) tests

%================================================%
R Programming/Nonparametric Methods
%- https://en.wikibooks.org/wiki/R_Programming/Nonparametric_Methods
This page deals with a set of non-parametric methods including the estimation of a cumulative distribution function (CDF), the estimation of probability density function (PDF) with histograms and kernel methods and the estimation of flexible regression models such as local regressions and generalized additive models.

For an introduction to nonparametric methods you can have a look at the following books or handout :

Nonparametric Econometrics: A Primer by Jeffrey S. Racine[1].
Li and Racine's handbook, Nonparametric econometrics[2].
Larry Wasserman All of Nonparamatric Statistics[3]
Contents  [hide] 
1 Empirical distribution function
2 Density Estimation
2.1 Histogram
2.2 Kernel Density Estimation
2.2.1 Examples
3 Local Regression
3.1 Examples
4 Generalized additive semiparametric models (GAM)
5 References
%---------------------------------------------%
Empirical distribution function[edit]
The easiest way to estimate the empirical CDF uses the rank() and the length() functions.
ecdf() computes the empirical cumulative distribution function.
ecdf.ksCI() (sfsmisc) plots the empirical distribution function with confidence intervals.
> N <- 1000
> x <- rnorm(N)
> edf <- rank(x)/length(x)
> plot(x,edf)
> plot(ecdf(x),xlab = "x",ylab = "Distribution of x")
> grid()
> library("sfsmisc")
> ecdf.ksCI(x1)


Density Estimation[edit]
Histogram[edit]
hist() is the standard function for drawing histograms. If you store the histogram as an object the estimated parameters are returned in this object.
> x <- rnorm(1000)
> hist(x, probability = T) # The default uses Sturges method.
> # Sturges, H. A. (1926) The choice of a class interval.
> # Journal of the American Statistical Association 21, 65–66. 
> hist(x, breaks = "Sturges", probability = T)
> 
> # Freedman, D. and Diaconis, P. (1981) On the histogram as a density estimator: L_2 theory.
> # Zeitschrift für Wahrscheinlichkeitstheorie und verwandte Gebiete 57, 453–476. 
> # (n^1/3 * range)/(2 * IQR).
> hist(x, breaks = "FD", probability = T)
> 
> # Scott, D. W. (1979). On optimal and data-based histograms. Biometrika, 66, 605–610. 
> # ceiling[n^1/3 * range/(3.5 * s)].
> hist(x, breaks = "scott", probability = T)
> 
> # Wand, M. P. (1995). Data-based choice of histogram binwidth.
> # The American Statistician, 51, 59–64. 
> library("KernSmooth")
> h <- dpih(x)
> bins <- seq(min(x)-h, max(x)+h, by=h)
> hist(x, breaks=bins, probability = T)
It is also possible to choose the break points.

> x <- rnorm(1000)
> hist(x, breaks = seq(-4,4,.1))
n.bins() (car package) includes several methods to compute the number of bins for an histogram.
histogram() (lattice)
truehist() (MASS)
hist.scott() (MASS) plot a histogram with automatic bin width selection, using the Scott or Freedman–Diaconis formulae.
histogram package.
Kernel Density Estimation[edit]
density() estimates the kernel density of a vector.
Choose the bandwidth selection method with bw.
Check the sensitivity of the bandwidth choice using adjust. The default is one. It is good practice to look at adjust=.5 and adjust=2.
> x <- rnorm(10^3)
> plot(density(x,bw = "nrd0", adjust = 1, kernel = "gaussian"), col = 1)
> lines(density(x,bw = "nrd0", adjust = .5, kernel = "gaussian"), col = 2)
> lines(density(x,bw = "nrd0", adjust = 2, kernel = "gaussian"), col = 3)
> legend("topright", legend = c("adjust = 1", "adjust = .5", "adjust = 2"), col = 1:3, lty = 1)
Choose the kernel function with kernel : "gaussian", "epanechnikov", "rectangular", "triangular", "biweight", "cosine", "optcosine".
> x <- rnorm(10^3)
> plot(density(x,bw = "nrd0", adjust = 1, kernel = "gaussian"), col = 1)
> lines(density(x,bw = "nrd0", adjust = 1, kernel = "epanechnikov"), col = 2)
> lines(density(x,bw = "nrd0", adjust = 1, kernel = "rectangular"), col = 3)
> lines(density(x,bw = "nrd0", adjust = 1, kernel = "triangular"), col = 3)
> legend("topright", legend = c("gaussian", "epanechnikov", "rectangular",  "triangular"), col = 1:4, lty = 1)
tkdensity() (sfsmisc) is a nice function which allow to dynamically choose the kernel and the bandwith with a handy graphical user interface. This is a good way to check the sensitivity of the bandwidth and/or kernel choice on the density estimation.
> x  <- rnorm(10^3)
> library("sfsmisc")
> tkdensity(x)
kde2d() (MASS) estimates a bivariate kernel density.
> N <- 1000
> x <- rnorm(N)
> y <- 1 + x^2 + rnorm(N)
> dd <-  kde2d(y,x) # estimate the bivariate kernel
> contour(dd) # plot the bivariate density
> image(dd) # another plot the bivariate density
Examples[edit]
Kernel density.svg
 
R-horsekick totals-density.svg
 
R-US state areas-density.svg
Local Regression[edit]
loess() is the standard function for local linear regression.
lowess() is similar to loess() but does not have a standard syntax for regression y ~ x .This is the ancestor of loess (with different defaults!).
ksmooth() (stats) computes the Nadaraya–Watson kernel regression estimate.
locpoly() (KernSmooth package)
npreg() (np package)
locpol computes local polynomial estimators
locfit local regression, likelihood and density estimation


Examples[edit]
NW cb16.png
Generalized additive semiparametric models (GAM)[edit]
gam() (gam)
gam() (mgcv)
> N <- 10^3
> u <- rnorm(N)
> x1 <- rnorm(N)
> x2 <- rnorm(N) + x1
> y <- 1 + x1^2 + x2^3 + u
> 
> library(gam)
> g1 <- gam(y ~ x1 + x2 ) # Standard linear model
> par(mfrow=c(1,2))
> plot(g1, se = T)
> 
> g1 <- gam(y ~ s(x1) + x2 ) # x1 is locally estimated
> par(mfrow=c(1,2))
> plot(g1, se = T)
> 
> g1 <- gam(y ~ s(x1) + s(x2) ) # x1 and x2 are locally estimated
> par(mfrow=c(1,2))
> plot(g1, se = T)
> 
> library(mgcv)
> g1 <- gam(y ~ s(x1) + s(x2) ) # x1 and x2 are locally estimated
> par(mfrow=c(1,2))
> plot(g1, se = T)


References[edit]
Jump up ↑ Jeffrey S. Racine Nonparametric Econometrics: A Primer http://socserv.mcmaster.ca/racine/ECO0301.pdf and at the R code examples http://socserv.mcmaster.ca/racine/primer_code.zip
Jump up ↑ Qi Li, Jeffrey S. Racine, Nonparametric econometrics, Princeton University Press - 2007
Jump up ↑ Wasserman, Larry, "All of nonparametric statistics", Springer (2007) (ISBN: 0387251456)
