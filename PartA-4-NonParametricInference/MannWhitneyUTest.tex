\documentclass[]{article}

%opening
\title{}
\author{}

\begin{document}

\maketitle

\section{WMann-Whitney U-test}




\subsection{Introduction}
The Mann-Whitney U test is used to compare differences between two independent groups when the dependent variable is either ordinal or continuous, but not normally distributed.
For example, you could use the Mann-Whitney U test to understand whether attitudes towards pay discrimination, where attitudes are measured on an ordinal scale, 
differ based on gender (i.e., your dependent variable would be "attitudes towards pay discrimination" 
and your independent variable would be "gender", which has two groups: "male" and "female"). 

Alternately, you could use the Mann-Whitney U test to understand whether salaries, measured on a continuous scale, differed 
based on education level (i.e., your dependent variable would be "salary" and your independent variable would be "educational level", which 
has two groups: "high school" and "university"). 

The Mann-Whitney U test is often considered the nonparametric alternative to the independent t-test although this is not always the case.

Unlike the independent-samples t-test, the Mann-Whitney U test allows you to draw different conclusions about your data depending on the assumptions 

you make about your data's distribution. These conclusions can range from simply stating whether the two populations differ through to 
determining if there are differences in medians between groups. These different conclusions hinge on the shape of the distributions of your 
data, which we explain more about later.

In our enhanced Mann-Whitney U test guide, which you can look inside here, we take you through all the steps required to understand when and
 how to use the Mann-Whitney U test, showing you the required procedures in SPSS, and how to interpret and report your output. 

In this "quick start" guide, we simply show you the basics of the Mann-Whitney U test using one of SPSS's procedures when the critical 
assumption of this test is violated. Before we show you how to do this, we explain the different assumptions that your data must meet 
in order for a Mann-Whitney U test to give you a valid result. We discuss these assumptions next, but for more detail, go to our enhanced Mann-Whitney U test guide here.

\subsection{Assumptions}
When you choose to analyse your data using a Mann-Whitney U test, part of the process involves checking to make sure that the data you want to analyse can actually be analysed using a Mann-Whitney U test. You need to do this because it is only appropriate to use a Mann-Whitney U test if your data "passes" four assumptions that are required for a Mann-Whitney U test to give you a valid result. We explain about these assumptions in more detail in our enhanced Mann-Whitney U test guide here. In practice, checking for these four assumptions just adds a little bit more time to your analysis, requiring you to click a few more buttons in SPSS when performing your analysis, as well as think a little bit more about your data, but it is not a difficult task.

Before we introduce you to these four assumptions, do not be surprised if, when analysing your own data using SPSS, one or more of these assumptions is violated (i.e., is not met). This is not uncommon when working with real-world data rather than textbook examples, which often only show you how to carry out a Mann-Whitney U test when everything goes well! However, don’t worry. Even when your data fails certain assumptions, there is often a solution to overcome this. First, let’s take a look at these four assumptions:

Assumption No.1: Your dependent variable should be measured at the ordinal or continuous level. Examples of ordinal variables include Likert items (e.g., a 7-point scale from "strongly agree" through to "strongly disagree"), amongst other ways of ranking categories (e.g., a 5-point scale explaining how much a customer liked a product, ranging from "Not very much" to "Yes, a lot"). Examples of continuous variables include revision time (measured in hours), intelligence (measured using IQ score), exam performance (measured from 0 to 100), weight (measured in kg), and so forth. You can learn more about interval and ratio variables in our article: Types of Variable.
Assumption No.2: Your independent variable should consist of two categorical, independent groups. Example independent variables that meet this criterion include gender (2 groups: male or female), employment status (2 groups: employed or unemployed), smoker (2 groups: yes or no), and so forth.
Assumption No.3: You should have independence of observations, which means that there is no relationship between the observations in each group or between the groups themselves. For example, there must be different participants in each group with no participant being in more than one group. This is more of a study design issue than something you can test for, but it is an important assumption of the Mann-Whitney U test. 


If your study fails this assumption, you will need to use another statistical test instead of the Mann-Whitney U test (e.g., a Wilcoxon signed-rank test). If you are unsure whether your study meets this assumption, you can use our Statistical Test Selector, which is part of our enhanced content.
Assumption No.4: A Mann-Whitney U test can be used when your two variables are not normally distributed. However, in order to know how to interpret the results from a Mann-Whitney U test, you have to determine whether your two distributions (i.e., the distribution of scores for both groups of the independent variable; for example, 'males' and 'females' for the independent variable, 'gender') have the same shape. To understand what this means, take a look at the diagram below:


In the two diagrams above, the distribution of scores for 'males' and 'females' have the same shape. In the diagram on the left, you cannot see the distribution of scores for 'males' (illustrated in blue on the diagram on the right) because the two distributions are identical (i.e., both distributions are identical, so they are 'on top of each other' in the diagram, with the blue-coloured male distribution underneath the red-coloured female distribution). 

However, in the diagram on the right, even though both distributions have the same shape, they have a different location (i.e., the distribution of one of the groups of the independent variable has higher or lower values compared to the second distribution – in our example, females have 'higher' values than males, overall).

When you analyse your own data, it is extremely unlikely that your two distributions will be identical, but they may have the same (or a 'similar') shape. If they do have the same shape, you can use SPSS to carry out a Mann-Whitney U test to compare the medians of your dependent variable (e.g., engagement score) for the two groups (e.g., males and females) of the independent variable (e.g., gender) you are interested in. However, if your two distribution have a different shape, you can only use the Mann-Whitney U test to compare mean ranks.

Therefore, when carrying out a Mann-Whitney U test, you must also use SPSS to determine whether your two distributions have the same shape or a different shape. This requires a few more procedures in SPSS, but it is an easy step-by-step process that we show you how to do in our enhanced Mann-Whitney U test guide. In this "quick start" guide, we show you how to carry out a Mann-Whitney U test assuming that your two distributions do not have a similar shape, such that you can only compare mean ranks and not medians.

You can check assumption 4 using SPSS. Before doing this, you should make sure that your data meets assumptions 1, 2 and 3, although you don't need SPSS to do this. Just remember that if you do not check assumption 4, you will not know whether you are correctly comparing mean ranks or medians, and the results you get when running a Mann-Whitney U test may not be valid. 

This is why we dedicate a number of sections of our enhanced Mann-Whitney U test guide to help you get this right. You can learn more about assumption 4 here and what you will need to interpret in the Assumptions section of our enhanced Mann-Whitney U test guide here.

In the Procedure section of this "Quick Start" guide, we illustrate the SPSS procedure to perform a Mann-Whitney U test assuming that your two distributions are not the same shape and you have to interpret mean ranks rather than medians. First, we set out the example we use to explain the Mann-Whitney U test procedure in SPSS.


The concentration of cholesterol (a type of fat) in the blood is associated with the risk of developing heart disease, such that higher concentrations of cholesterol indicate a higher level of risk, and lower concentrations indicate a lower level of risk. If you lower the concentration of cholesterol in the blood, your risk for developing heart disease can be reduced. Being overweight and/or physically inactive increases the concentration of cholesterol in your blood. Both exercise and weight loss can reduce cholesterol concentration. However, it is not known whether exercise or weight loss is best for lowering blood cholesterol concentration. Therefore, a researcher decided to investigate whether an exercise or weight loss intervention is more effective in lowering cholesterol levels. 

To this end, the researcher recruited a random sample of inactive male individuals that were classified as overweight. This sample was then randomly split into two groups: Group 1 underwent a calorie-controlled diet and Group 2 undertook an exercise training programme. In order to determine which treatment programme was more effective, cholesterol concentrations were compared between the two groups at the end of the treatment programmes.


In SPSS, we entered the scores for cholesterol concentration, our dependent variable, under the variable name Cholesterol. Next, we created a grouping variable, called Group, which represented our independent variable. Since our independent variable had two groups - 'diet' and 'exercise' - we gave the diet group a value of "1" and the exercise group a value of "2". If you do not label your two groups, SPSS will not be able to distinguish between them and the Mann-Whitney U test will not run properly.

Note: There are two different procedures in SPSS to run a Mann-Whitney U test: a new and legacy procedure. How we have explained the data setup above relates to the legacy procedure (and the new procedure when your dependent variable is continuous), which is what we take you through in the Procedure section next. We mention this because if you are using the new procedure, you have to make changes to your data setup if your dependent variable is ordinal (i.e., as opposed to being continuous). We explain how to do this in our enhanced Mann-Whitney U test guide here

In our enhanced Mann-Whitney U test guide, we show you all the steps required to correctly enter data into SPSS to run a Mann-Whitney U test for both the new and legacy procedures discussed in the note above. We show you this data setup process here.

If you read assumption No. 4 earlier, you'll know that the SPSS procedure when analysing your data using a Mann-Whitney U test is different depending on the shape of the two distributions of your independent variable. In our example, where our dependent variable is cholesterol concentration, Cholesterol, we are referring to the two distributions of the independent variable, Group (i.e., the distribution of scores for Group 1 – the 'diet' group – and Group 2 – the 'exercise' group). In the 10 steps below, we show you how to analyse your data using a Mann-Whitney U test in SPSS when these two distributions have a different shape, and therefore, you have to compare the mean ranks of your dependent variable rather than medians.

To use SPSS to determine whether your two distributions have the same or different shapes, or if you want to know how to use SPSS to carry out a Mann-Whitney U test when your two distribution have the same shape, such that you need to compare medians rather than mean ranks, you will need to access the Procedures section of our enhanced Mann-Whitney U test guide (learn more here). Furthermore, the 10 steps below also show you how to carry out a Mann-Whitney U test using the legacy procedure in SPSS. As we explained earlier, there are two different procedures in SPSS to run a Mann-Whitney U test: a new and legacy procedure. We recommend the new procedure if your two distributions have the same shape because it is a little easier to carry out, but the legacy procedure is fine if your two distributions have different shapes. We show you the new and legacy procedures in our enhanced Mann-Whitney U test guide (learn more here).

\end{document}