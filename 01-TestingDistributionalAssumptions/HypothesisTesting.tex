\section{Hypothesis Testing}


Hypothesis testing is a common practice in science that involves conducting tests and experiments to see if a proposed explanation for an observed phenomenon works in practice. A hypothesis is a tentative explanation for some kind of observed phenomenon, and is an important part of the scientific method. The scientific method is a set of steps that is commonly employed by those in scientific fields to give scientific explanations for various phenomena.

Any tentative explanation can be referred to as a hypothesis if it can be submitted to hypothesis testing. There are, however, a set of guidelines for an explanation to be considered a true scientific hypothesis. The first major point is testability; a scientific hypothesis must be able to proceed to the stage of hypothesis testing to be considered a scientifically legitimate hypothesis. It is generally suggested that a hypothesis be relatively simple, though this is not always possible. Hypotheses must also be able to explain the phenomena under any set of conditions; if a hypothesis can only explain a phenomenon in one set of conditions, it is generally considered unacceptable.

Hypotheses are generally considered useful only if they are likely to improve on the current body of knowledge on a subject and pave the way for greater knowledge to be acquired in the future. Also, a hypothesis is generally not acknowledged if it defies other commonly recognized knowledge. If a hypothesis meets all of these requirements, it will typically proceed to the hypothesis testing phase.

In hypothesis testing, the testers seek to discover evidence that either validates or disproves a given hypothesis. Usually, this involves a series of experiments being conducted in many different conditions. If the hypothesis does not stand up to the tests in all conditions, something is usually wrong with the hypothesis and a new one must be formed to take the new information into account. The new hypothesis is submitted to the same hypothesis testing. If it passes and is not proven wrong, it can eventually be considered a scientific theory or law, though nothing in science can be proven to be absolutely true.

One common method of hypothesis testing is known as statistical hypothesis testing, and typically deals with large quantities of data. Experiments and tests are conducted and the data is collected. If the data collected shows that it is unlikely that the results occurred by chance, it is considered statistically significant and can be used to support a hypothesis.


\section{Degrees of Freedom}

Degree of freedom (df) is a concept most used in statistics and physics. In both cases it tends to define limits of a system and position or size of what is being analyzed, so that it can be visually represented. Definition of df in both fields is related, but not quite the same.

In physics, degree of freedom positions objects or systems, and each degree references a position in time, space or in other measurements. Df could be used synonymously with the term coordinate, and it usually means independent coordinates of the fewest number. The actual degree of freedom is based on the system being described in phase space or in all the potential types of space a system inhabits simultaneously. Every single part of phase space the system takes up can be considered a df, which helps to define the full realities of the system being considered.

From a statistical standpoint, degree of freedom defines distributions of populations or samples and is encountered when people begin to study inferential statistics: hypothesis testing and confidence intervals. As with the scientific definition, df in statistics describes shape or aspects of sample or population depending on data. Not all drawn representations of distributions have a degree of freedom measurement. The common standard normal distribution is not defined by degrees; instead, it will be the same bell-shaped curve in all instances.

A similar distribution to standard normal is student-t. The student-t is defined in part by degree of freedom in the formula n-1, where n is sample size. This means that were variables from the distribution to be picked one by one, all but the last one could be chosen freely. There is no choice but to take the very last one and no freedom to choose any other variable at that point. Therefore one variable is not free; it’s like having to pick the last tile out of a bag during a Scrabble® game where there is no choice but to choose that letter.

Different distributions like the F and the chi-square have different definitions of degree of freedom, and some even use more than one df in definition. The issue gets confusing because df definition is linked to type of test performed and isn’t the same with the various parametric (based on parameters) and non-parametric (not based on parameters) tests. Essentially, it won’t always be n-1. Goodness of fit or contingency table testing may use the chi-square distribution with different df than that which evaluates single variable hypothesis testing of the variance or standard deviation.

What is important to remember is that each time degree of freedom is used to define a distribution, it changes it. It still may have certain characteristics that are unchanging, but size and appearance vary. When people are drawing representations of distributions, particularly two of the same distributions that have a different df, they’re advised to make them look different in size to convey that df is not the same.



\section{Statistical significance}

Statistical significance is a mathematical tool used to determine whether the outcome of an experiment is the result of a relationship between specific factors or due to chance. Statistical significance is commonly used in the medical field to test drugs and vaccines and to determine causal factors of disease. Statistical significance is also used in the fields of psychology, environmental biology, and any other discipline that conducts research through experimentation.

Statistics are the mathematical calculations of numeric sets or populations that are manipulated to produce a probability of the occurrence of an event. Statistics use a numeric sample and apply that number to an entire population. For the sake of example, we might say that $80\%$ of all Americans drive a car. It would be difficult to question every American about whether or not they drive a car, so a random number of people would be questioned and then the data would be statistically analyzed and generalized to account for everyone.

In a scientific study, a hypothesis is proposed, then data is collected and analyzed. The statistical analysis of the data will produce a number that is statistically significant if it falls below $5\%$, which is called the confidence level. In other words, if the likelihood of an event is statistically significant, the researcher can be $95\%$ confident that the result did not happen by chance.

Sometimes, when the statistical significance of an experiment is very important, such as the safety of a drug meant for humans, the statistical significance must fall below $3\%$. In this case, a researcher could be $97\%$ sure that a particular drug is safe for human use. This number can be lowered or raised to accommodate the importance and desired certainty of the result being correct.

Statistical significance is used to reject or accept what is called the null hypothesis. A hypothesis is an explanation that a researcher is trying to prove. The null hypothesis holds that the factors a researcher is looking at have no effect on differences in the data. Statistical significance is usually written, for example, $t=.02, p<.05$. Here, "t" stands for the statistic test score and "p<.05" means that the probability of an event occurring by chance is less than $5\%$. These numbers would cause the null hypothesis to be rejected, therefore affirming that the alternative hypothesis is true.

Here is an example of a psychological hypothesis using statistical significance: It is hypothesized that baby girls smile more than baby boys. In order to test this hypothesis, a researcher would observe a certain number of baby girls and boys and count how many times they smile. At the end of the observation, the numbers of smiles would be statistically analyzed.

Every experiment comes with a certain degree of error. It is possible that on the day of observation all the boys were abnormally grumpy. The statistical significance found by the analysis of the data would rule out this possibility by 95\% if t=.03. In this case, the null hypothesis that baby girls do not smile more than baby boys would be rejected, and with 95\% certainty, the researcher could say that girls smile more than boys.
