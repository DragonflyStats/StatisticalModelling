\documentclass{beamer}

\usepackage{framed}
\usepackage{graphicx}
\usepackage{amsmath}
\usepackage{amssymb}

\begin{document}
	
	\begin{frame}
		\frametitle{MANOVA}
		Multivariate Measures: In most of the statistical programs used to calculate MANOVAs
		there are four multivariate measures: Wilks’ lambda, Pillai's trace, Hotelling-Lawley trace
		and Roy’s largest root. The difference between the four measures is the way in which
		they combine the dependent variables in order examine the amount of variance in the
		data. 
	\end{frame}
	%========================================================== %
	\begin{frame}
		\frametitle{MANOVA}
		Wilks’ lambda demonstrates the amount of variance accounted for in the dependent
		variable by the independent variable; the smaller the value, the larger the difference
		between the groups being analyzed. 1 minus Wilks’ lambda indicates the amount of
		variance in the dependent variables accounted for by the independent variables.
	\end{frame}
	%========================================================== %
	\begin{frame}
		\frametitle{MANOVA} Pillai's
		trace is considered the most reliable of the multivariate measures and offers the greatest
		protection against Type I errors with small sample sizes. 
	\end{frame}
	%========================================================== %
	\begin{frame}
		\frametitle{MANOVA}
		Pillai's trace is the sum of the
		variance which can be explained by the calculation of discriminant variables. It calculates
		the amount of variance in the dependent variable which is accounted for by the greatest
		separation of the independent variables. The Hotelling-Lawley trace is generally 
				converted to the Hotelling’s T-square. 
	\end{frame}
	%========================================================== %
	\begin{frame}
		\frametitle{MANOVA}
		Hotelling’s T is used when the independent
		variable forms two groups and represents the most significant linear combination of the
		dependent variables. Roy’s largest root, also known as Roy’s largest eigenvalue, is
		calculated in a similar fashion to Pillai's trace except it only considers the largest
		eigenvalue (i.e. the largest loading onto a vector). As the sample sizes increase the values
		produced by Pillai’s trace, Hotelling-Lawley trace and Roy’s largest root become similar.
		As you may be able to tell from these very broad explanations, the Wilks’ lambda is the
		easiest to understand and therefore the most frequently used measure.
	\end{frame}

	%========================================================== %
	\begin{frame}
		\frametitle{MANOVA}
		Multivariate F value: This is similar to the univariate F value in that it is representative of
		the degree of difference in the dependent variable created by the independent variable.
		However, as well as being based on the sum of squares (as in ANOVA) the calculation
		for F used in MANOVAs also takes into account the covariance of the variables. 
		\end{frame}
		
		%========================================================== %
		\begin{frame}
			\frametitle{MANOVA}	
			Wilks' lambda is a test statistic used in multivariate analysis of variance
			(MANOVA) to test whether there are differences between the means of
			identified groups of subjects on a combination of dependent variables. 
			
		\end{frame}

		%========================================================== %
		\begin{frame}
			\frametitle{MANOVA}
			
			\begin{itemize}
				\item For
				example, in the paper above, the authors test whether the mean score of two
				groups, graduates and diplomates, is the same across eight constructs
				simultaneously. 
				\item Thus, they are considering eight dependent variables and
				comparing the mean of this combination for two groups.
			\end{itemize}
		\end{frame}
		%========================================================== %
		\begin{frame}
			\frametitle{MANOVA}
			Wilks' lambda performs, in the multivariate setting, with a combination of
			dependent variables, the same role as the F-test performs in one-way analysis
			of variance. Wilks' lambda is a direct measure of the proportion of variance in
			the combination of dependent variables that is unaccounted for by the
			independent variable (the grouping variable or factor). 
		\end{frame}
		%========================================================== %
		\begin{frame}
			\frametitle{MANOVA}
			If a large proportion
			of the variance is accounted for by the independent variable then it suggests
			that there is an effect from the grouping variable and that the groups (in this
			case the graduates and diplomates) have different mean values.
		\end{frame}

		%========================================================== %
		\begin{frame}
			\frametitle{MANOVA}
			Wilks' lambda statistic can be transformed (mathematically adjusted) to a
			statistic which has approximately an F distribution. This makes it easier to
			calculate the P-value. Often authors will present the F-value and degrees of
			freedom, as in the above paper, rather than giving the actual value of Wilks'
			lambda.
		\end{frame}
		%========================================================== %
		\begin{frame}
			\frametitle{MANOVA}
			There are a number of alternative statistics that can be calculated to
			perform a similar task to that of Wilks' lambda, such as Pillai's trace criterion
			and Roy's gcr criterion; however, Wilks' lambda is the most widely used.
			Everitt \& Dunn (1991) and Polit (1996) provide more detail about the use
			and interpretation of Wilks' lambdaWilks' lambda is a test statistic used in multivariate analysis of variance
			(MANOVA) to test whether there are differences between the means of
			identified groups of subjects on a combination of dependent variables. 
			
		\end{frame}
		%========================================================== %
		\begin{frame}
			\frametitle{MANOVA}
			For
			example, in the paper above, the authors test whether the mean score of two
			groups, graduates and diplomates, is the same across eight constructs
			simultaneously. Thus, they are considering eight dependent variables and
			comparing the mean of this combination for two groups.
		\end{frame}
		%========================================================== %
		\begin{frame}
			\frametitle{MANOVA}
			Wilks' lambda performs, in the multivariate setting, with a combination of
			dependent variables, the same role as the F-test performs in one-way analysis
			of variance. Wilks' lambda is a direct measure of the proportion of variance in
			the combination of dependent variables that is unaccounted for by the
			independent variable (the grouping variable or factor). If a large proportion
			of the variance is accounted for by the independent variable then it suggests
			that there is an effect from the grouping variable and that the groups (in this
			case the graduates and diplomates) have different mean values.
		\end{frame}
		%========================================================== %
		\begin{frame}
			\frametitle{MANOVA}
			Wilks' lambda statistic can be transformed (mathematically adjusted) to a
			statistic which has approximately an F distribution. This makes it easier to
			calculate the P-value. Often authors will present the F-value and degrees of
			freedom, as in the above paper, rather than giving the actual value of Wilks'
			lambda.
			
		\end{frame}
		%========================================================== %
		\begin{frame}
			\frametitle{MANOVA}
			There are a number of alternative statistics that can be calculated to
			perform a similar task to that of Wilks' lambda, such as Pillai's trace criterion
			and Roy's gcr criterion; however, Wilks' lambda is the most widely used.
			Everitt \& Dunn (1991) and Polit (1996) provide more detail about the use
			and interpretation of Wilks' lambda
		\end{frame}
		
	\end{document}
