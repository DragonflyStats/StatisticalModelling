\documentclass[a4paper,12pt]{article}
%%%%%%%%%%%%%%%%%%%%%%%%%%%%%%%%%%%%%%%%%%%%%%%%%%%%%%%%%%%%%%%%%%%%%%%%%%%%%%%%%%%%%%%%%%%%%%%%%%%%%%%%%%%%%%%%%%%%%%%%%%%%%%%%%%%%%%%%%%%%%%%%%%%%%%%%%%%%%%%%%%%%%%%%%%%%%%%%%%%%%%%%%%%%%%%%%%%%%%%%%%%%%%%%%%%%%%%%%%%%%%%%%%%%%%%%%%%%%%%%%%%%%%%%%%%%
\usepackage{eurosym}
\usepackage{vmargin}
\usepackage{amsmath}
\usepackage{graphics}
\usepackage{epsfig}
\usepackage{subfigure}
\usepackage{fancyhdr}
%\usepackage{listings}
\usepackage{framed}
\usepackage{graphicx}

\setcounter{MaxMatrixCols}{10}
%TCIDATA{OutputFilter=LATEX.DLL}
%TCIDATA{Version=5.00.0.2570}
%TCIDATA{<META NAME="SaveForMode" CONTENT="1">}
%TCIDATA{LastRevised=Wednesday, February 23, 2011 13:24:34}
%TCIDATA{<META NAME="GraphicsSave" CONTENT="32">}
%TCIDATA{Language=American English}

\pagestyle{fancy}
\setmarginsrb{20mm}{0mm}{20mm}{25mm}{12mm}{11mm}{0mm}{11mm}
\lhead{Stata} \rhead{mhodds}
\chead{}

\begin{document}

\section*{Mantel-Haenszel Methods - \texttt{mhodds}}

\begin{verbatim}

  Overview:  The "mhodds" command is used to obtain summaries and inference
             for case-control or cross-sectional designs.  It calculates an
             estimate of the odds ratio (disease & exposure) for each level 
             of a stratifying variable.  
             
             A test for the common odds ratio, and an estimate of the common 
             odds ratio are given.  In addition, a test of homogeneity 
             of the odds ratios is computed.

             One key advantage of "mhodds" is the ability to control for
             multiple stratifying variables.


  Usage:     "mhodds Dvar Evar, by(Svar)"
             Where - Dvar is the disease variable (1=disease, 0=control)
                     Evar is the exposure variable (1=exposed, 0=unexposed)
                     Svar is the stratifying variable (multiple levels)


  Summaries: The "mhodds" command returns the stratum specific odds ratio,
             a test for each stratum specific odds ratio, a confidence interval
             for each stratum specific odds ratio, and an adjusted, or common
             odds ratio estimate, test, and confidence interval.
             
\end{verbatim}
\newpage
\section*{Mantel-Haenszel Methods - \texttt{cc}}

\begin{verbatim}

Overview:  The "cc" command is used to obtain summaries and inference
             for case-control or cross-sectional designs.  Using the
             "by(Svar)" option allows Mantel-Haenszel estimates to be 
             obtained.


  Usage:     "cc Dvar Evar, by(Svar)"
             Where - Dvar is the disease variable (1=disease, 0=control)
                     Evar is the exposure variable (1=exposed, 0=unexposed)
                     Svar is the stratifying variable (multiple levels)


  Summaries: The "cc" command returns the stratum specific odds ratio and
             confidence interval, the Mantel-Haenszel weights, the crude
             odds ratio and the adjusted odds ratio.  In addition, both 
             a test of homogeneity of odds ratios, and the Mantel-Haenszel
             test for the common odds ratio are computed.

\end{verbatim}
\end{document}             
