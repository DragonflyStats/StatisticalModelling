%- http://idatasciencer.com/2015/04/regression-exercise-for-learners/

Regression: Exercises for Learners
25. Apr. 2015/MLR, R, Regression, Tutorial/No Comments
A collection of Exercises in Regression Analysis that a Learner must know “How to”

In the table below, we have results from an environmental engineering study of a certain chemical reaction.

There are 21 concentrations of separately prepared solutions recorded at different times (three measurements at each of seven times).

In the last column we have natural logarithms of the concentrations for your reference.

The data can also be downloaded in CSV from This Link

You will use the data from this table to complete certain exercises mentioned in the last section after data table.
Solution Number
(i)
Time (Xi)
(hrs)
Concentration (Yi)
(mg/ml)
Ln of Concentration
(lnYi)
1

6

0.029

-3.540

2

6

0.032

-3.442

3

6

0.027

-3.612

4

8

0.079

-2.538

5

8

0.072

-2.631

6

8

0.088

-2.430

7

10

0.181

-1.709

8

10

0.165

-1.802

9

10

0.201

-1.604

10

12

0.425

-0.856

11

12

0.384

-0.957

12

12

0.472

-0.751

13

14

1.130

0.122

14

14

1.020

0.020

15

14

1.249

0.222

16

16

2.812

1.034

17

16

2.465

0.902

18

16

3.099

1.131

19

18

3.614

1.285

20

18

3.402

1.224

21

18

3.913

1.364

Use “R” to complete the following exercises:

 

Exercise One
Generate separate graphs of:

Concentration (Y) vs. Time (X)
Natural Logarithm of Concentration (lnY) vs. Time (X)
 

Exercise Two
Equations and Plotting

Using the output from exercise one, obtain the following:

The estimated equation of the straight-line (degree 1) regression of Y on X
The estimated equation of the quadratic (degree 2) regression of Y on X
The estimated equation of the straight-line (degree 1) regression of lnY on X
Plots of each of these fitted equations on their respective scatter diagrams.
 

Exercise Three
Determine and Compare

Determine and compare the proportions of the total variation in Y explained by the straight-line regression on X and by the quadratic regression on X
 

Exercise Four
F-Tests

Carry out F-tests for the significance of the straight-line regression of Y on X
Carry out an overall F-test for the significance of the quadratic regression of Y on X and a test for the significance of the addition of x2 to the model
For the straight-line regression of lnY on X, carry out F-tests for the significance of the overall regression
 

Exercise Five
Determine and Compare

What proportion of the variation in lnY is explained by the straight-line regression of lnY on X?
Compare this result with that obtained in Exercise Three for the quadratic regression of lnY on X.
 

Exercise Six
Examine and Discuss

Use comment box on this page (below) and explain your thoughts on the following:

A fundamental assumption in regression analysis is variance homoscedasticity. By examining the scatter diagrams constructed in Exercises One & Two, state why taking natural logarithms of the concentrations helps with regard to the assumption of variance homogeneity.
Do you think the straight-line regression of lnY on X is better for describing this set of data than the quadratic regression of Y on X?
Considering the overall table, what key assumption about the data would be in question if, instead of 21 different solutions, there were only 3 different solutions, each of which was analyzed at the seven different time points?
