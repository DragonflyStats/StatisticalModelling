\section{ONE SAMPLE: THE WILCOXON TEST}
Just as is the case for the sign test, the Wilcoxon test can be used to test a null hypothesis concerning the
value of the population median. Because the Wilcoxon test considers the magnitude of the difference between
each sample value and the hypothesized value of the median, it is a more sensitive test than the sign test. On the
other hand, because differences are determined, the values must be at least at the interval scale. No assumptions
are required about the form of the population distribution.

%=====================================================================================%
\begin{itemize}
\item The null and alternative hypotheses are formulated with respect to the population median as either a one-
sided or two-sided test. The difference between each observed value and the hypothesized value of the median
is determined, and this difference, with arithmetic sign, is designated d:d ¼ (X  Med 0). If any difference is
equal to zero, the associated observation is dropped from the analysis and the effective sample size is thereby
reduced. 
\item The absolute values of the differences are then ranked from lowest to highest, with the rank of 1
assigned to the smallest absolute difference. When absolute differences are equal, the mean rank is assigned to
the tied values. \item Finally, the sum of the ranks is obtained separately for the positive and negative differences. 
\end{itemize}

%=================================================================================================================%
The
smaller of these two sums is the Wilcoxon T statistic for a two-sided test. In the case of a one-sided test, the
smaller sum must be associated with the directionality of the null hypothesis. Appendix 10 identifies the critical
values of T according to sample size and level of significance. For rejection of the null hypothesis, the obtained
value of T must be smaller than the critical value given in the table

%=================================================================================================================%
\section*{Example}
Use the Wilcoxon test with respect to the null hypothesis and the data in Problem 17.2, above, and
compare your solution to the one obtained in Problem 17.2.
H0: Med ¼ 80

%====================================================================================== %
In order to reject the null hypothesis H0: Med ¼ 80 for this upper-tail test, it follows that the differences
d ¼ (X   Med0) must be predominantly positive, for negative differences would represent positive support for
the null hypothesis. Therefore, for this upper-tail test the sum of the ranks for the negative differences must be
the smaller sum. Referring to Table 17.2, we see that such is in fact the case, and the value of the Wilcoxon test
statistic is T ¼ 7:5.

%====================================================================================== %
The null hypothesis is to be tested at the 5 percent level of significance. Appendix 10 indicates that for
n ¼ 11 (the effective sample size) the critical value of T for a one-tail test at the 5 percent level of
significance is T ¼ 14. Because the obtained, value of T is smaller than the critical value, the null hypothesis
is rejected at the 5 percent level of significance, and we conclude that the median units assembled with the
new system is greater than 80 units. 


%====================================================================================== %
This is the same conclusion as with the use of the sign test in the
preceding section, in which the P value for the test was P ¼ 0:0328. However, reference to Appendix 10 for
the Wilcoxon test indicates that for n ¼ 11 and a ¼ 0:025 the critical value is T ¼ 11. Therefore, the null
hypothesis could be rejected at the 2.5 percent level of significance with the Wilcoxon test, but not with the
sign test. This observation is consistent with our earlier observations that the Wilcoxon test is the more
sensitive of the two tests.
H1: Med . 80

%====================================================================================== %
The sample data are repeated in Table 17.2, which is the worktable for the Wilcoxon test. In this
table note that for the fourth sampled workshift the number of units assembled happened to be
exactly equal to the hypothesized value of the population median. Therefore, this observation was
dropped from the analysis, resulting in an effective sample size of n ¼ 11. Also note that the absolute
value of d for the first and second sampled workshifts is the same, and therefore the mean rank of 4.5
was assigned to each of these sampled workshifts (in lieu of ranks 4 and 5). The next rank assigned is
then rank 6.

%====================================================================================== %
\end{document}