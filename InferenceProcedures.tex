
{Today's Class}

\begin{itemize}
	\item Inference tests .
	\begin{itemize}
		\item Kolmogorov Smirnov (K-S) test.
		\item Grubbs Test
		\item Anderson Darling Test
		\item Shapiro Wilk Test
		\item Chi Square test.
	\end{itemize}
	\item I will do the most important test, the ``t - test", in the next class.
\end{itemize}




%------------------------------------------
{K-S test: one sample test}

\begin{itemize}
	\item One sample test - tests whether or not a data set comes from a specified distribution.
	\item The distribution is specified in the argument, by passing as an argument the name of a function associated with that distribution, but not the quantile function.
	\item To specify the normal distribution use either ``prnorm",``dnorm" or ``rnorm", but not ``qnorm".
	\item The null hypothesis is that the data set is normally distributed (or other specified distribution).
	\item The alternative is that it is not normally distributed.
	\item The mean and standard deviation must be specified, to distinguish from the standard normal distribution.
	
\end{itemize}

%------------------------------------------
{K-S test: one sample test}
Generate two random sets of data values: x and y.
\begin{verbatim}
> ks.test(x,"rnorm", mean(x),sd(x))

One-sample Kolmogorov-Smirnov test

data:  x
D = 0.143, p-value = 0.938
alternative hypothesis: two-sided

\end{verbatim}

The p-value is very high, greater than 1\%. We fail to reject the null. The data is indeed normally distributed.


%------------------------------------------
{K-S test: two sample test}

\begin{itemize}
	\item Two sample test - tests whether or not two data sets come from the same distribution.
	\item The null is that they do come from the same distribution. The alternative is that they don't.
\end{itemize}

\begin{verbatim}
> ks.test(x,y)

Two-sample Kolmogorov-Smirnov test

data:  x and y
D = 0.25, p-value = 0.869
alternative hypothesis: two-sided
\end{verbatim}
Again a high p-value. We can conclude that they do come from the same distribution.


%------------------------------------------

{Other Inference tests}

The nortest package includes common tests for normality of distribution. Similarly the outliers package contains well known inference tests for outliers.

\begin{itemize}
	\item Grubbs test for outliers [outliers].
	\item Dixon Test for outliers [outliers].
	\item Anderson Darling test for normality [nortest].
\end{itemize}
A boxplot is useful in conjunction with such tests.



%------------------------------------------

{Grubbs test}
\begin{verbatim}

> library(outliers)
> grubbs.test(x)
Grubbs test for one outlier
data:  x
G = 2.4180, U = 0.4202, p-value = 0.02405
alternative hypothesis:
lowest value 3.51 is an outlier
\end{verbatim}

%------------------------------------------

{Grubbs test: conclusion}
\begin{itemize}
	\item The null hypothesis is that lowest value (highest in other cases) is not an outlier.
	\item The alternative hypothesis that it is an outlier.
	\item The p-value (0.02405) is less than 5\% (usual value for $\alpha$).
	\item Therefore we reject the null hypothesis.
	\item Lowest value is an outlier.
\end{itemize}

%------------------------------------------
{Anderson-Darling normality test}
\begin{verbatim}
> ad.test(x)

Anderson-Darling normality test

data:  x
A = 0.3859, p-value = 0.3325
\end{verbatim}


%------------------------------------------
{The Shapiro Wilk normality test}
The Shapiro Wilk is another commonly used procedure used to test normality.
Again, the null hypothesis is that the data set is normally distributed.
\begin{verbatim}
> x=rnorm(40)
> shapiro.test(x)

Shapiro-Wilk normality test

data:  x
W = 0.9601, p-value = 0.1689
\end{verbatim}
Here, the P-value is greater than 1\%. We fail to reject the null hypothesis.

%------------------------------------------

{A-D test: conclusion}
\begin{itemize}
	\item The null hypothesis is that data set is normally distributed.
	\item The alternative hypothesis is that it is not normally distributed.
	\item The p-value (0.3325) is greater than than 1\%.
	\item Therefore we fail to reject the null hypothesis.
\end{itemize}


%---------------------------------------------------------------------
{Chi Square test}

Chi Square test are used when one wants to check whether a sample comes from some type of population,
or when one wants to check that two samples are from the same population.

\begin{itemize}
	\item Goodness of fit tests.
	\item Testing for association in contingency tables.
\end{itemize}

The $R$ command for chi squared tests is chisq.test().



%---------------------------------------------------------------------
{Chi-Square test}
\begin{itemize}
	\item Goodness of fit test is used to test whether a sample comes from a specified distribution.
	\item The sample is univariate.
	\item $R$ assumes a null hypothesis that each outcome is equally likely ( uniformly distributed).
	\item chisq.test(x)
	\item Can specify the distribution of expected values under the null hypothesis for other cases using the `p' parameter.
	\item chisq.test(x,p)
\end{itemize}

%---------------------------------------------------------------------
{Chi Square}
\begin{itemize}
	\item Example: In an experiment of 100 trials, the number of times that each of the four possible outcomes
	occured was recorded.\\
	\begin{tabular}{|c|c|c|c|c|}
		\hline
		Event & A & B & C & D \\\hline
		Outcomes & 59 & 20 & 11 & 10 \\
		\hline
	\end{tabular}
	
	\item It was presupposed that each outcome was equally likely \\(i.e. The null hypothesis is $H_{0} : P_{A} = P_{B} = P_{C} = P_{D}$.)\\
	\begin{tabular}{|c|c|c|c|c|}
		\hline
		Event & A & B & C & D \\\hline
		Expected & 25 & 25 & 25 & 25 \\
		\hline
	\end{tabular}
	\item Under the null hypothesis, any deviations from the expected values are attributable to random error.
	
\end{itemize}

%---------------------------------------------------------------------
{Chi Square tests}


\begin{verbatim}
> chisq.test(c(59,20,11,10))

Chi-squared test for given probabilities

data:  c(59, 20, 11, 10)
X-squared = 64.08, df = 3,
p-value = 7.891e-14
\end{verbatim}

The p-value indicates that we should reject the hypothesis that there is an equal likely outcomes.


%---------------------------------------------------------------------
{Chi Square test}
\begin{itemize}
	\item If you do not want equal proportions, you need to give a set of
	proportions for each cell (using the `p' parameter).
	\item Genetic theory predicts that certain fruit flies
	will fall into four categories in proportions 9:3:3:1.
	\item Data showed counts of 59, 20, 11 and 10.
\end{itemize}
\begin{verbatim}
> chisq.test(c(59,20,11,10),
p=c(9/16,3/16,3/16,1/16))

Chi-squared test for given probabilities

data:  c(59, 20, 11, 10)
X-squared = 5.6711, df = 3, p-value = 0.1288
\end{verbatim}

{Chi Square test}
\begin{itemize}
	\item We would not reject the theoretical hypothesis with these data.
	\item Deviations from expected values are attributed to random error.
\end{itemize}


%--------------------------------------------------------------------
{Contingency Tables}

\begin{itemize}
	\item We also use the Chi square test for testing association in two way
	contingency tables.
	\item Contingency tables: Outcomes are categorized into rows and columns.
	\item This can be used to test the differences between several groups.
	\item The null hypothesis is that there is not differences between the groups.
	\item The alternative is that there is difference between the groups.
	\item Independence or association?
\end{itemize}




%--------------------------------------------------------------------

{Chi Square: Contingency table}

Example: Test the association between smoking and exercise.

\begin{verbatim}
> library(MASS)     # load the MASS package
> tbl = table(survey$Smoke, survey$Exer)
> tbl               # the contingency table

Freq None Some
Heavy    7    1    3
Never   87   18   84
Occas   12    3    4
Regul    9    1    7


\end{verbatim}


{Chi Square: Contingency table}
Test the hypothesis whether the students smoking habit is independent of their exercise level at .05 significance level.
\begin{verbatim}
> chisq.test(tbl)

Pearson's Chi-squared test

data:  tbl
X-squared = 5.4885, df = 6, p-value = 0.4828

Warning message:
In chisq.test(tbl) : Chi-squared
approximation may be incorrect

\end{verbatim}




%---------------------------------------------------------------------
{Chi Square: Contingency table}
\begin{itemize}
	\item We have applied the chisq.test() function to the contingency table tbl, and found the p-value to be 0.4828.
	\item We fail to reject the null hypothesis.
	
	
	\item The warning message found in the solution above is due to the small cell values in the contingency table. \item To avoid such warning, we could combine the second and third columns of tbl.
\end{itemize}




\section{Test for Equality of Variance and Means}

\begin{itemize}
\item Test for Equality of Test (\texttt{var.test()})
\item Welch Two Sample \emph{t-}test (\texttt{t.test()})
\item Independent Two Sample \emph{t-}test (\texttt{t.test(var.equal=TRUE)})

\end{itemize}

\subsection{Bartlett's test for Homogeneity of Variances}
 

Equal variances across samples is called homogeneity of variances. Bartlett's test is used to test if multiple samples have equal variances. 

Some statistical tests, such as the analysis of variance, assume that variances are equal across groups or samples.  The Bartlett test can be used to verify that assumption.

\begin{itemize}
\item The null hypothesis is that each of the samples have equal variance.
\item The alternative hypothesis states that at least one sample has a significantly different variance.
\end{itemize}

%----------------------------------------------------------------------------------------------------------------- %
\newpage

\section{Inference Procedures}
\subsection{Confidence Interval }
A confidence interval gives an estimated range of values which is likely to include an unknown population parameter, the estimated range being calculated from a given set of sample data. If independent samples are taken repeatedly from the same population, and a confidence interval calculated for each sample, then a certain percentage (confidence level) of the intervals will include the unknown population parameter. 

Confidence intervals are usually calculated so that this percentage is $95\%$, but we can produce $90\%$, $99\%$, $99.9\%$ (or whatever) confidence intervals for the unknown parameter. The width of the confidence interval gives us some idea about how uncertain we are about the unknown parameter. A very wide interval may indicate that more data should be collected before anything very definite can be said about the parameter.
\subsection{Power }
The power of a statistical hypothesis test measures the test's ability to reject the null hypothesis when it is actually false - that is, to make a correct decision. In other words, the power of a hypothesis test is the probability of not committing a type II error. It is calculated by subtracting the probability of a type II error from 1, usually expressed as: 
\[\mbox{Power} = 1 - \mbox{P(type II error) } = 1- \beta \]The maximum power a test can have is 1, the minimum is 0. Ideally we want a test to have high power, close to 1.

\section{Single Sample Inference Procedures}
If we have a single sample we might want to answer several
questions:
\begin{itemize}
	\item What is the mean value? \item Is the mean value
	significantly different from current theory? (Hypothesis test)
	\item What is the level of uncertainty associated with our
	estimate of the mean value? (Confidence interval)
\end{itemize}

\begin{itemize}
	\item (Last week : confidence interval for a mean) \item Revision:
	For large samples ($n > 30$) and/or if the population standard
	deviation ($\sigma$) is known, the usual test statistic is given
	by: \[Z =\frac{\bar{X} - \mu}{SE(\bar{X})}\]
	
	\item $S.E.(\bar{X}) = { \sigma \over \sqrt{n}} $ or ${s \over \sqrt{n}}$. 
	\item For small samples, use the $t-$distribution with $n-1$ degrees of freedom.
	\item Critical value from tables.
	\item Compare test statistics and critical values.
\end{itemize}

To ensure that our analysis is correct we need to check for
outliers in the data (i.e. boxplots) and we also need to check
whether the data are normally distributed or not.

\begin{framed}
	\begin{verbatim}
	> t.test(X,mu=10)
	
	One Sample t-test
	
	data:  X 
	t = 14.1421, df = 4, p-value = 0.0001451
	alternative hypothesis: true mean is not equal to 10 
	95 percent confidence interval:
	10.08037 10.11963 
	sample estimates:
	mean of x 
	10.1 
	\end{verbatim}
\end{framed}


%--------------------------------------------------------------------------------------------------%


%--------------------------------------------------------------------------%
\newpage
%section 9 Inference Procedures




\subsection{Hypothesis test of Proportion}
This procedure is used to assess whether an assumed proportion is supported by evidence. For two tailed tests, the null hypothesis states that the population proportion  π has a specified value, with the alternative stating that π has a different value. 

The hypotheses are typically as follows:   

\begin{itemize}
	\item[Ho] : $\pi = 0.50$
	\item[Ha] : $\pi \neq 0.50$
\end{itemize}

\subsubsection{Example}
A manufacturer is interested in whether people can tell the difference between a new formulation of a soft drink and the original formulation. The new formulation is cheaper to produce so if people cannot tell the difference, the new formulation will be manufactured. 

A sample of 100 people is taken. Each person is given a taste of both formulations and asked to identify the original. Sixty-two percent of the subjects correctly identified the new formulation. Is this proportion significantly different from $50\%$? 

The first step in hypothesis testing is to specify the null hypothesis and an alternative hypothesis. In testing proportions, the null hypothesis is that $\pi$, the proportion in the population, is equal to 0.5. The alternate hypothesis is $\pi \neq 0.5$. 

The computed p-values is compared to the pre-specified significance level of $5\%$. Since the p-value (0.0214) is less than the significance level of 0.05, the effect is statistically significant. 

\begin{verbatim}
> prop.test(62,100,0.5)

1-sample proportions test with continuity correction

data:  62 out of 100, null probability 0.5 
X-squared = 5.29, df = 1, p-value = 0.02145
alternative hypothesis: true p is not equal to 0.5 
95 percent confidence interval:
0.5170589 0.7136053 
sample estimates:
p 
0.62 
\end{verbatim}

Since the effect is significant, the null hypothesis is rejected. It is concluded that the proportion of people choosing the original formulation is greater than 0.50. 

This result might be described in a report as follows: 

\begin{quote}
	The proportion of subjects choosing the original formulation (0.62) was significantly greater than 0.50, with p-value = 0.021.
\end{quote}  

%----------------------------------------------------%


\subsubsection{Two Sample Tests}


All of the previous hypothesis tests and confidence intervals can be
extended to the two-sample case.

The same assumptions apply, i.e. data are normally distributed in
each population and we may want to test if the mean in one
population is the same as the mean in the other population, etc.

Normality can be checked using histograms, boxplots and Q-Q
plots as before. The Anderson-Darling test can be used on
each group of data also.


%------------------------------------------------------%
\subsubsection{Implementation}

This can be carried out in R by hand:

\footnotesize \begin{verbatim}
>obs.vals <- matrix(c(43,9,44,4), nrow=2, byrow=T)
>row.tots <- apply(obs.vals, 1, sum)
>col.tots <- apply(obs.vals, 2, sum)
>exp.vals <- row.tots%o%col.tots/sum(obs.vals)
>TS <- sum((obs.vals-exp.vals)^2/exp.vals)
>TS
>[1] 1.777415
\end{verbatim}\normalsize


%------------------------------------------------------%

%----------------------------------------------------%


\subsubsection{Two Sample Tests}


All of the previous hypothesis tests and confidence intervals can be
extended to the two-sample case.

The same assumptions apply, i.e. data are normally distributed in
each population and we may want to test if the mean in one
population is the same as the mean in the other population, etc.

Normality can be checked using histograms, boxplots and Q-Q
plots as before. The Anderson-Darling test can be used on
each group of data also.


%------------------------------------------------------%
\subsubsection{Implementation}

This can be carried out in R by hand:

\footnotesize \begin{verbatim}
>obs.vals <- matrix(c(43,9,44,4), nrow=2, byrow=T)
>row.tots <- apply(obs.vals, 1, sum)
>col.tots <- apply(obs.vals, 2, sum)
>exp.vals <- row.tots%o%col.tots/sum(obs.vals)
>TS <- sum((obs.vals-exp.vals)^2/exp.vals)
>TS
>[1] 1.777415
\end{verbatim}\normalsize

