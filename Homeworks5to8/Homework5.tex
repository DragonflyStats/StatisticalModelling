\documentclass[a4paper,12pt]{article}
%%%%%%%%%%%%%%%%%%%%%%%%%%%%%%%%%%%%%%%%%%%%%%%%%%%%%%%%%%%%%%%%%%%%%%%%%%%%%%%%%%%%%%%%%%%%%%%%%%%%%%%%%%%%%%%%%%%%%%%%%%%%%%%%%%%%%%%%%%%%%%%%%%%%%%%%%%%%%%%%%%%%%%%%%%%%%%%%%%%%%%%%%%%%%%%%%%%%%%%%%%%%%%%%%%%%%%%%%%%%%%%%%%%%%%%%%%%%%%%%%%%%%%%%%%%%
\usepackage{eurosym}
\usepackage{vmargin}
\usepackage{amsmath}
\usepackage{graphics}
\usepackage{framed}
\usepackage{enumerate}
\usepackage{epsfig}
\usepackage{subfigure}
\usepackage{fancyhdr}

\setcounter{MaxMatrixCols}{10}
%TCIDATA{OutputFilter=LATEX.DLL}
%TCIDATA{Version=5.00.0.2570}
%TCIDATA{<META NAME="SaveForMode"CONTENT="1">}
%TCIDATA{LastRevised=Wednesday, February 23, 201113:24:34}
%TCIDATA{<META NAME="GraphicsSave" CONTENT="32">}
%TCIDATA{Language=American English}

\pagestyle{fancy}
\setmarginsrb{20mm}{0mm}{20mm}{25mm}{12mm}{11mm}{0mm}{11mm}
\lhead{MA4505} \rhead{Kevin O'Brien} \chead{SPSS Homework 5 } %\input{tcilatex}

\begin{document}
	\begin{framed}
		\begin{itemize}
			\item Deadline : Midnight Friday Week 9
			\item Submit by Email
			\item Subject Line : "\textbf{MA405 Homework 5 Submission}" \\ \textit{(Dont not write anything else)}
		\end{itemize}
	\end{framed}
	
\subsection*{SPSS Homework 5  Task 1: Visualizing Categorical Data}


\begin{itemize}
	\item Open your data set in the \texttt{Butter} Folder.
	\item There is likely to be a fair bit of SPSS output for this particular exercise. This is probably the only time that will happen.
    \item Include a Narrative or introduction at the start.
\end{itemize}


%Homework 5A: Descriptive Statistics (1\%)
%Data Set: \texttt{Butter}
\begin{itemize}
\item Include the SPSS output in your submission.
\item Descriptive Statistics of "fat content" (i.e. The Fat Variable)
\item Descriptive Statistics of "fat content" for each Process
\item Descriptive Statistics of "fat content" for each Factory
\item Compare outcomes for each group,  commenting upon
\begin{itemize}
	\item  mainly : Mean, Median, 
\item also   : Max, Min, Range,  Standard Deviation
\end{itemize}
\end{itemize}
\bigskip
\subsubsection*{Part A: Frequency Table of Animal Types (3 Marks)}
Data Set :  Market Research

\textit{(Remark: The Market Research Data Set got mixed up with the Animals Data Sets. I am going to leave it be - using Market Research, which contains information on Animals - So we are market researching on Geese and Hamsters.)}
\begin{enumerate}[(i)]
	\item No Narrative or Introduction is required here at all.
	\item Frequency Table of Animal Types
	\item Try to add as much information into the table as possible, using percentages and cumulative percentages etc
\end{enumerate}


\subsubsection*{Part B: Frequency Tables and Cross Tabulations }
\begin{verbatim}
Data Sets : MarketResearchUpdated 
(if you were doing this on the old version, just continue. 
Make sure you get something similar to below)

Create the following Frequency Tables
- Frequency Table for Sex.
- Frequency Table for Each Age Group.
- Frequency Table for Each Brand Preference.
- Accompany each table with a Pie-Chart, or Bar Chart.
\end{verbatim}
For each combination of pairs of categorical Variables:

\begin{itemize}
\item Create a Crosstabulation with Count and Row Percentages only (\textbf{Type 1}).
\item Create a Crosstabulation with Count and Column Percentage only (\textbf{Type 2}).
\item For the sake of consistency: 
\begin{itemize}
	\item Always Pick the "Sex" variable for "Rows"
	\item Always pick the "Brand Preference" Variable for "Columns"
\end{itemize}
     
\item You should have 6 cross-tabulations in total.\\ (3 combinations $\times$ 2 Types)
\end{itemize}
\subsubsection*{Part C: Bar Chart (aka Bar Plot)}
\begin{enumerate}[(i)]
	\item Create a Bar Chart of Animal Ages
\end{enumerate}
%===============================================================%
\bigskip
\subsection*{SPSS Homework 5  Task 2 : Descriptive Statistics }

\begin{itemize}
\item Explain what case-wise differences are.
\item Open your data set in the \texttt{BeforeAfter}.
\item Compute the case-wise differences (After-Before)
\item Provide the Descriptive Statistics for Case-Wise Difference.
\item State the significance value and confidence intervals.
\end{itemize}


\end{document}