UNIVERSITY OF LIMERICK
OLLSCOIL LUIMNIGH






COLLEGE OF INFORMATICS AND ELECTRONICS


MODULE CODE:				MS4215

MODULE TITLE:				Advanced Data Analysis

SEMESTER:					Autumn 2007/2008

LECTURER:					Prof. E. Murphy

DURATION:					1½  Hours

INSTRUCTIONS TO CANDIDATES:	Attempt ALL 3 QUESTIONS 
						All questions carry equal marks

						Exam: 40%
						Lab Based Practical: 40%
						Project: 20%





Question 1
Matrix Notation

(a)	y =  x  β  +  

Prove that D () =  2

Where D() is the dispersion matrix


(b) Rewrite the linear (scalar) model 

yi = βo + β1  xi + I   in matrix format.

And using the results from Part (a) above, or otherwise.  Prove any 2 of the following:

V   = 

V  

Cov   =  


(c) Using the results from Part (b) prove that the variance of the mean value of 
      y corresponding to x = xo in the simple linear model is 

 2  {  }
        



Hint Part (b) 			1	 x1
				1	 x2
				1 	 xn
		




Q2
The structural model, for k treatments each with n observations, in one way ANOVA may be represented by  yij = μ + αi + εij
(a) You are required to

(i) Derive the least squares estimator for αi   Please indicate clearly any assumptions made in your calculations.
(ii) Explain the terms  (i) Fixed and (ii) Random effects



(b) Consider a factorial experiment with n observations per cell and 3 factors A, B
and C in p x q x r in a fully factorial design.
(i) Write out the structural model for the above design
(ii) Outline the rules for deriving the expected values for crossed designs involving random effects
(iii)      Using the rules outlined above, write out the expected values of the Mean    Square for factor  A and interactions AB, AC and ABC respectively.

(c) The table below represents the ANOVA table for three random factors, each at two levels.

Source
S. Squares
Mean Square
Expected M Sq
A
15
15
*
B
10
10

C
  5
  5

AB
  8
  8
*
AC
10
10
*
BC
  8
  8

ABC
  4
  4
*
Error
40
  5

Total
100



.

Detail how you would use the information from part (b) above  to develop an appropriate QUASI F ratio to test the hypothesis that the variance (A) = 0.


Q3

The vice president for sales in a large corporation wished to assess the impact of market promotions on gross sales. A random sample of 15 quarters market promotions and gross sales were analysed and the resulting Minitab analysis is reproduced below.




Regression Analysis: Sales versus Market 

The regression equation is
Sales = 105 + 0.871 Market





S = 12.1532   R-Sq = 37.7%   


Analysis of Variance

Source          DF      SS      MS     F      P
Regression       *       *       *      *  0.015
Residual Error   *       *       *
Total           14  3079.6



Predictor    Coef  SE Coef     T      P
Constant   105.05    16.56  6.35  0.000
Market     0.871       *      *      *

Sum of Squares for X (SSx = 1527)

You are required to 

(a)	Complete the blanks in the above tables. Please show all calculations clearly

(b) 	What is the relationship between the missing values for T and F 


(c) 	If  the continuous variable sales  were  reclassified into the binomial variable 1= success and 2 = failure , what difference ,if any , would this information have on how you would analyse the data set.

(d)	Write a brief note on each of the following elements of a GLM.

(i) Random component
(ii) Systematic Component
(iii) Link Function.

(e)	How do ordinary Regression and ANOVA models relate to GLMs, identify the link function where appropriate





