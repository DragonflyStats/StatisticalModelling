\documentclass[pdf,default,slideColor,colorBG]{prosper}



% define a new font called goodfont
\def\goodfont{\usefont{T1}{pcr}{b}{n}\fontsize{36pt}{40pt}\selectfont\green}
\renewcommand{\familydefault}{\rmdefault}
\renewcommand{\rmdefault}{cmr}
\parindent 0pt
\parskip 5pt


\begin{document}


%-----------------------------------------------
\begin{slide}{Statistical Inference(2)}
\begin{itemize}
\item Test for proportions (\texttt{prop.test()})
    \begin{itemize}
    \item One sample procedure
    \item Two sample procedure
    \end{itemize}
\item Correlation test (\texttt{cor.test()})
\item Shapiro-Wilk test (\texttt{shapiro.test()})
\end{itemize}
\end{slide}


%-----------------------------------------------

\begin{slide}{\texttt{prop.test()}}
We can also perform a hypothesis test for a population
proportion, \textit{p}. The \texttt{R} function to carry out such a test is
prop.test. The parameters are
\begin{itemize}
\item \texttt{x} : number of success ( value or vector)
\item \texttt{n} : sample size (value or vector)
\item (sample proportion : $\hat{p} = x/n$)
\item \texttt{p} : Null hypothesis value is 50\%($ H_0: p = 0.5 $)
\item \texttt{conf.level} : confidence level (default is 95\%)
\item \texttt{alternative} : relationship in alt. hypothesis (default: two.tailed)
\end{itemize}

\end{slide}

%-----------------------------------------------

\begin{slide}{\texttt{prop.test() }- one sample}
Consider this simple example where the population proportion (in other words, the probability of success) is assumed to be 50\%
\begin{itemize}
\item 280 successes from 400 trials.
\item default null hypothesis($ H_0: p = 0.5 $)
\item default alt. hypothesis($ H_1: p \neq 0.5 $)
\end{itemize}

\begin{verbatim}

> prop.test(280,400)
\end{verbatim}
\end{slide}
%----------------------------------------
\begin{slide}{prop.test - one sample}
\begin{verbatim}

> prop.test(280,400)

    1-sample proportions test
    with continuity correction

data:  280 out of 400, null probability 0.5
X-squared = 63.2025, df = 1,
p-value = 1.865e-15
alternative hypothesis: 
    true p is not equal to 0.5
95 percent confidence interval:
 0.6520722 0.7440176
sample estimates:
  p
0.7
\end{verbatim}

%----------------------------------------

\end{slide}

\begin{slide}{prop.test - one sample}
\begin{itemize}
\item can access components (just like a data frame)
\item Consider both the p-value 
\item and the confidence interval
\end{itemize}

\begin{verbatim}

> names(prop.test(280,400))
[1] "statistic"   "parameter"   "p.value"
[4] "estimate"    "null.value"  "conf.int"
[7] "alternative" "method"      "data.name"
>
> prop.test(280,400)$p.value
[1] 1.865115e-15
\end{verbatim}
\end{slide}
%----------------------------------------------

\begin{slide}{prop.test - one sample}
\begin{itemize}
\item p.values is <0.0001. 
\item reject the null hypothesis.
\item $p$ (i.e. 0.5) not in confidence interval.
\item validates conclusion.
\end{itemize}


\end{slide}

%----------------------------------------

\begin{slide}{prop.test - null hypothesis}
Test the null hypothesis that true proportion is 0.65 (i.e. $H_{0}: p = 0.65$)
\begin{verbatim}

> prop.test(280,400,p=0.65)
.....
data:  280 out of 400, null probability 0.65
X-squared = 4.1786, df = 1, p-value = 0.04094
alternative hypothesis: 
    true p is not equal to 0.65
95 percent confidence interval:
 0.6520722 0.7440176

\end{verbatim}
We reject the null hypothesis (p-value and CI).

\end{slide}

%----------------------------------------------------------
\begin{slide}{prop.test - confidence interval}
We can specify the confidence (and significance) level.
\begin{verbatim}
> prop.test(280,400,conf.level=0.99)
.....
data:  280 out of 400, null probability 0.5
X-squared = 63.2025, df = 1, 
p-value = 1.865e-15
alternative hypothesis:
    true p is not equal to 0.5
99 percent confidence interval:
 0.6368118 0.7565247
.....
\end{verbatim}

\end{slide}
%---------------------------------------------
\begin{slide}{prop.test - alternative hypotheses}
We can specify one tail tests by using the `alternative' parameter.
\begin{itemize}
\item alternative ="greater" 
\item alternative ="less"
\end{itemize}
\begin{verbatim}

> prop.test(280,400,p=0.65,
        alternative="greater")
        
> prop.test(280,400,p=0.65,
        alternative="less")
\end{verbatim}
\end{slide}

%---------------------------------------------
\begin{slide}{prop.test - alternative hypotheses}
\begin{verbatim}

> prop.test(280,400,p=0.65,
        alternative="greater")
.....
data:  280 out of 400, null probability 0.65
X-squared = 4.1786, df = 1, p-value =
0.02047
alternative hypothesis: 
    true p is greater than 0.65
95 percent confidence interval:
 0.659785 1.000000
\end{verbatim}
We reject the null hypothesis (p-value and CI).

\end{slide}
%----------------------------------------
\begin{slide}{prop.test(): two samples}

To compare two samples, create a vector containing the number of successes in each sample, and a vector containing the sample sizes. 
This will test the null hypothesis $H_0:p_{1}=p_{2}$.
\begin{verbatim}
>x.vector <- c( 83, 70 )
>n.vector<- c( 86, 82 )
>prop.test(xvec, nvec)
\end{verbatim}
\end{slide}
%----------------------------------------
\begin{slide}{prop.test(): two samples}
\begin{verbatim}
> prop.test(x.vector, n.vector)

        2-sample test for equality 

data:  x.vector out of n.vector
X-squared = 5.1155, df = 1, p-value = 0.02371
alternative hypothesis: two.sided
95 percent confidence interval:
 0.01377774 0.20913775
sample estimates:
   prop 1    prop 2
0.9651163 0.8536585 
\end{verbatim}
\end{slide}

%----------------------------------------
\begin{slide}{prop.test(): multiple samples}
Can use prop.test() for more than 2 samples. The null hypothesis in such a case would be all true proportions are equal. The alternative hypothesis is that at least one has a different true mean.
\begin{itemize}
\item Study on cigarette smokers (Fleiss (1981), p. 139)
\item H0: The null hypothesis is that the four populations from which
the patients were drawn have the same true proportion of smokers.
\item H1: The alternative is that this proportion is different in at
least one of the populations.
\end{itemize}
\begin{verbatim}
>smokers  <- c( 83, 90, 129, 70 )
>patients <- c( 86, 93, 136, 82 )
>prop.test(smokers, patients)
\end{verbatim}
\end{slide}



%----------------------------------------
\begin{slide}{prop.test() - Example}


Example: A manufacturer claims that the proportion of
defective items produced is approximately 4\%. A random sample of
size 50 is taken, 3 of which are defective. Is the manufacturer's
claim justified?

\begin{verbatim}
>prop.test(x=3, n=50, p = 0.04,
alternative = "two.sided",
conf.level = 0.95)
\end{verbatim}
\end{slide}

%---------------------------------------------------
\begin{slide}{prop.test() - Example}

\begin{verbatim}
        1-sample proportions test

data:  3 out of 50, null probability 0.04
X-squared = 0.1302, df =1, p-value = 0.7182
alternative hypothesis:
 true p is not equal to 0.04
95 percent confidence interval:
 0.01562459 0.17541874
...........
\end{verbatim}
\end{slide}

%---------------------------------------------------
\begin{slide}{cor.test() }
We can test correlation in bivariate data using the cor.test() function.
This test can be applied to all correlation coefficients.\\ We will only consider the default type, Pearson's product moment correlation coefficient.\\
The default null hypothesis for cor.test() is that the correlation coefficient is zero.
The default alternative is that the correlation coefficient is not zero.
\begin{verbatim}
> A<-c(4,5,6,2,5,1,2,7,9)
> B<-c(4,7,5,3,6,4,2,7,2)
> cor(A,B)
[1] 0.2415396
>cor.test(A,B)
\end{verbatim}

\end{slide}
%----------------------------------------------------------
\begin{slide}{cor.test() }



\begin{verbatim}
> cor.test(A,B)
 Pearson's product-moment correlation

data:  A and B
t = 0.6586, df = 7, p-value = 0.5312
alternative hypothesis: 
    true correlation is not equal to 0
95 percent confidence interval:
 -0.5033206  0.7804655
sample estimates:
      cor
0.2415396
\end{verbatim}

\end{slide}
%----------------------------------------------------------
\begin{slide}{cor.test() }
\begin{itemize}
\item Conclusions: p-value and CI validate `fail to reject' conclusion.
\item We can specifythe confidence level and alternative hypothesis as per previous procedures, but null value must be zero.
\item Can specify other correlation types using `method'.
\end{itemize}

\end{slide}

%----------------------------------------------------------
\begin{slide}{Shapiro Wilk Test }
\begin{itemize}
\item Used to determine if a data set is normally distributed
\item Null hypothesis is that it is normally distributed.
\item Alt. hypothesis is that it is not normally distributed.
\item Advise using with graphical complement.
\item Output return test statistic (W) and p-value.
\end{itemize}
\begin{verbatim}
> shapiro.test(A)

        Shapiro-Wilk normality test

data:  A
W = 0.961, p-value = 0.8085
\end{verbatim}
\end{slide}
\end{document}

