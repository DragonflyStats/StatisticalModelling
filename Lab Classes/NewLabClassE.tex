download and install an R package

install.pacakges("MethComp")
install.pacakges("chemCal")
install.packages("mcreg")
Packages are collections of R functions, data, and compiled code in a well-defined format. The directory where packages are stored is called the library. R comes with a standard set of packages. 

Others are available for download and installation.

\begin{framed}
\begin{verbatim}
library()   # see all packages installed 
search()    # see packages currently loaded
\end{verbatim}
\end{framed}

\subsection{Adding Packages}
You can expand the types of analyses you do be adding other packages. A complete list of contributed packages is available from CRAN.
Follow these steps:

Download and install a package (you only need to do this once).
To use the package, invoke the library(package) command to load it into the current session. 
(You need to do this once in each session, unless you customize your environment to automatically load it each time.)

On MS Windows:
Choose Install Packages from the Packages menu.
Select a CRAN Mirror. (e.g. Ireland)
Select a package. (e.g. chemCal)

Then use the library(package) function to load it for use. (e.g. library(chemCal))

You can install a package directly using the \texttt{install.packages()} command. Once installed you can directly call the library, using the ibrary command. 

Important - some packages require the more recent version of R to be installed.

\begin{framed}
\begin{verbatim}
install.packages("MethComp")
install.packages("chemCal")

library(MethComp)
library(chemCal)
\end{verbatim}
\end{framed}

chemCal: Calibration functions for analytical chemistry

chemCal provides simple functions for plotting linear calibration functions and estimating standard errors for measurements according to the Handbook of Chemometrics and Qualimetrics: Part A by Massart et al. There are also functions estimating the limit of detection (LOD) and limit of quantification (LOQ). The functions work on model objects from - optionally weighted - linear regression (lm) or robust linear regression (rlm from the MASS package).


calplot
inverse.predict
lod

\begin{framed}
\begin{verbatim}
# get data set and fit linear model
data(massart97ex3)
m <- lm(y ~ x, data = massart97ex3)

calplot(m)
\end{verbatim}
\end{framed}
%==================================================================%

\begin{framed}
\begin{verbatim}
# This is example 7 from Chapter 8 in Massart et al. (1997)
# get data set and fit linear model
data(massart97ex1)
m <- lm(y ~ x, data = massart97ex1)

inverse.predict(m, 15) 
inverse.predict(m, 90) 
inverse.predict(m, rep(90,5)) 

\end{verbatim}
\end{framed}

%==================================================================%

\begin{framed}
\begin{verbatim}

data(din32645)
m <- lm(y ~ x, data = din32645)
lod(m)
# The critical value (decision limit, German Nachweisgrenze) can be obtained
# by using beta = 0.5:
lod(m, alpha = 0.01, beta = 0.5)

\end{verbatim}
\end{framed}



For each of the three three fits models, determine the analysis of variance table.
( In R code ouput the "total" row is not included. Disregard this row for today.

%===================================================%

To extract the residuals from a fitted model, we use the resid() command, specifying the name of the model.

Recall last week we fitted several models for the Cheeses data set ( run the labD-Setup script again).

Fit2res <- resid(Fit2)

#  Type in "Fit2res" to get a sense of the data.

compute the variance of the Fit2 residauls (use the var command)

plot(Fit2) 
# Hit Return after inspecting each screen

Four plots are automatically presented but in fact there are six altogether. 
These plots will indicate influential points. 
To specify an individual plot to be presented, type each of the following command.


plot(Fit2, which=c(1)) 
plot(Fit2, which=c(2)) 
plot(Fit2, which=c(3)) 
plot(Fit2, which=c(4)) 
plot(Fit2, which=c(5)) 
plot(Fit2, which=c(6)) 

%===================================================%
Residuals
Heteroscedascity ( Non-Constant Variance)
Influence, outliers, Cooks Distance Leverage,
The Funnel Effect
%====================================================%
Weighted Regression
Robust Regression

%====================================================%
Deming Regression ( 20 Minutes )
%==================================================%
import CSV

create Data Frame

%==================================================%
Orthonormal (Deming) Regression

Residuals
plot the residuals
compute the variance of the residuals

plot(Fitall)


analysis of variance

aov

(anova is a distinct command for determining)
