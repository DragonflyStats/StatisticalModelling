Course Outline 2014-2015
 
All students take the Base Module, which is taught on Mondays and Wednesdays during the first 12 week semester, before Christmas.  Students must then take two elective modules to complete the course. For the first six weeks of the second semester (after Christmas) there will be a module entitled Introduction to Regression (Tuesdays and Thursdays). This will be followed by a six-week module on the Design and Analysis of Experiments (also Tuesdays and Thursdays) and a module on Time Series (Mondays and Wednesdays) - these two modules will be taught from weeks 7 to 12. Note that the Time Series module is a bit more technical than the others and may not be a suitable choice for students with a weak mathematical background.
 
 
 
ST7001: Base Module 

Michaelmas Term (Semester 1): Monday and Wednesday: (18.00-20.00)
 
Lecturer: Eamonn Mullins (Eamonn.mullins@tcd.ie)
 

Topics Covered: 

• Data summaries and graphs
 • Statistical models
 • Sampling distributions: confidence intervals and tests
 • Simple comparative experiments: t-tests, confidence intervals, design issues
 • Counted data: confidence intervals and tests for proportions
 • Cross-classified frequency data: chi-square tests
 • Introduction to Regression Analysis
 • Introduction to Analysis of Variance
 • Statistical computing laboratory 

 
 
 
 
ST7002: Introduction to Regression 

Hilary Term (Semester 2) Weeks 1-6: Tuesday and Thursday (18.00-20.00)
 
Lecturer: Professor John Haslett (John.Haslett@tcd.ie)
 

  

Topics covered:
 
• Statistical versus deterministic relationships 
• Simple linear regression model: assumptions, model fitting, estimation of coefficients and their standard errors
 • Confidence intervals and statistical significance tests on model parameters
 • Prediction intervals
 • Analysis of variance in regression: F-tests, r-squared
 • Model validation: residuals, residual plots, normal plots, diagnostics
 • Multiple regression analysis - short introduction
 • Statistical computing laboratories 

 
 
 
 
ST7003: Design and Analysis of Experiments 

Hilary Term (Semester 2): Weeks 7-12: Tuesday and Thursday (18.00 - 20.00) 

Lecturer: Dr. Michael Stuart (Michael.Stuart@tcd.ie)
 

This module is concerned with the design of data collection exercises for the assessment of the effects of changes in factors associated with a process and the analysis of the data subsequently produced.
 
In order to assure that the experimental changes caused the observed effects, strict conditions of control of the process must be adhered to. Specifically, the conditions under which the experimentation is conducted must be as homogeneous as possible with regard to all extraneous factors that might affect the process, other than the experimental factors that are deliberately varied.
 
The simplest experiments involve comparison of process results when a single factor is varied over two possible conditions.  When more than two factors are involved, issues regarding the most efficient choice of combinations of factor conditions and ability to detect interactions between factors become important.  With many factors and many possible experimental conditions for each factor, the scale of a comprehensive experimental design becomes impractical and suitable strategies for choosing informative subsets of the full design are needed.
 
The analysis of data resulting from well designed experiments is often very simple and graphical analysis can be very effective.  Standard statistical significance tests may be used to assure that apparent effects are real and not due simply to chance process variation.  Confidence intervals are used in estimating the magnitude of effects.  In cases with more complicated experimental structure, a more advanced technique of statistical inference, Analysis of Variance, may be used.
 
Minitab is well equipped to assist both with design set up and with analysis of subsequent data, both graphical and formal.  There will be two laboratory sessions involving the use of Minitab.
 
Case studies and illustrations from a range of substantive areas will be discussed
 
Learning Outcomes
 
On successful completion of this module, students should be able to
 compare and contrast observational and experimental studies,
describe and explain the roles of control, blocking, randomisation and replication in experimentation,
explain the advantages of statistical designs for multifactor experiments,
describe and explain the genesis of basic experimental design structures,
implement  and interpret the analysis of variance for a selection of experimental designs,
describe the models underlying the analysis of variance for a selection of experimental designs,
produce and interpret graphs for data summary and model diagnostics,
provide outline descriptions of more elaborate designs and data analyses,
describe and discuss strategic issues involved in the design and implementation of experiments.
 
Syllabus
 
Specific topics addressed in this module include:
 
The need for experiments 
experimental and observational studies
 cause and effect
 control
Basic design principles for experiments
 Control
 Blocking (pairing)
 Randomisation
 Replication
 Factorial structure
 
Standard designs
 Randomised blocks
 Two-level factors
 Multi-level factors
 
Split units
 
Analysis of experimental data
 Exploratory data analysis
 Parameter estimation and significance testing
 Analysis of variance
 Statistical models, fixed and random effects
 Model validation, diagnostics
 Software laboratories
 
Review topics
 Block structure and treatment structure
 Repeated measures 
Analysis of Covariance
 Clinical trials
 Response surface designs
 Robust designs
 Non-Normal errors
 Strategies for Experimentation
 
Assessment One 3-hour examination
 
References
 
Mullins, E., Statistics for the Quality Control Chemistry Laboratory, Royal Society of Chemistry, 2003, particularly Chapters 4-5, 7-8.  Detailed coverage of much of the module, in a specific context.
Mead, R., The design of experiments: statistical principles for practical applications, Cambridge University Press, 1988. Comprehensive text, with extensive discussion of fundamentals.
Box, G.E.P., Hunter, J.S. and Hunter, W.G., Statistics for Experimenters, 2nd. ed., Wiley, 2005.  Includes many gems of wisdom from these masters of the genre, though not a course text.
 Daniel, C., Applications of Statistics to Industrial Experimentation, Wiley, 1976.  Includes many gems of wisdom from this master of the genre, using methodology appropriate for an industrial setting.
 Robinson, G.K., Practical Strategies for Experimenting, Wiley, 2000.  A comprehensive review of the non-statistical aspects of planning and conducting experiments and interpreting and using their results.
  
 

ST7004: Aspects of Survey Design 

Lecturer: Dr. Myra O'Regan (Myra.ORegan@tcd.ie)
 
This module will focus on data collection using questionnaires.   The following topics will be addressed on the course.
•        The survey process
•        Sample selection
•        Measurement issues
•        Designing paper and web based questionnaires

 This module will include a computer laboratory element that will show students how to develop online questionnaires.
 
The model will be assessed on the basis of a project.  Each student will be required to carry out a survey of their 
choice and write a detailed report on the development of the questionnaire.   The number of students is limited to 30.  Should more than 30 students apply, participants will be chosen at random.
  
 
 
 
ST7005: Time Series Analysis (N.B. THIS MODULE IS NOT AVAILABLE IN 2014/2015)
 
Hilary Term (Semester 2): Weeks 7-12: Monday and Wednesday (18.00 - 20.00) 


Lecturer: Dr. Rozenn Dahyot (Rozenn.Dahyot@scss.tcd.ie)
 
Aims:
 
Several methods of forecasting will be examined, including exponential smoothing and its Hot-Winters extension, auto-regression, moving average, and further regression based methods that take into account seasonal trends of lagged variables. The module will be practical, and will involve every student in extensive analysis of case study material for a variety of time series data.
 
Learning Outcomes:
 
When students have successfully completed this module they should be able to:
 Define and describe the different patterns that can be found in times series and propose the methods that can be used for their analysis.
Program, analyse and select the best model for forecasting.
Interpret output of data analysis performed by a computer statistics package.
 
Syllabus:
 Introduction to times series
Regression
Autoregressive Models
Data Transformations
Modelling Seasonality
Decomposition
Exponential Smoothing
RMSE and MAPE performance measures
Holt-Winters Models for Seasonality
Autocorrelation
Brief Introduction to ARIMA
 
Bibliography: Forecasting - Methods and Applications, S. Makridakis, S.C. Wheelwright and R. J. Hyndman, Wiley
 
Website: http://www.scss.tcd.ie/Rozenn.Dahyot/
 
 
 
Please note: 
Elective modules may vary from year to year. The department reserves the right to cancel a module if there is not sufficient demand. 

Note that the information regarding lectures and lecture times, while correct when published, may be subject to change due to unforeseen circumstances. 
