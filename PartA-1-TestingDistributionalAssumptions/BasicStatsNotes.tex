
\subsection{Numerical Methods}
The only way of describing qualitative data is using graphical methods. 
The first step in describing quantitative data is using graphical methods i.e. get a picture of your data. 

Since quantitative data are numeric, however, we can also use numeric methods i.e. calculate a set of numbers that convey a good mental picture of the frequency distribution. 

There are two main numerical descriptive measures :

\begin{enumerate}
\item Measure of centrality i.e. measure of the centre of the distribution
\item Measure of dispersion i.e. the spread, dispersion of the data
\end{enumerate}
When we know the middle of our distribution and how spread out it is about the middle, we have two numbers which create a concise numerical summary of the data. 

The measure of centrality and dispersion we use depends on what our data looks like i.e. the shape of the distribution.

%------------------------------------------------%
\subsection{Percentiles}
A percentile is defined as a point below which a certain per cent of the observations lie e.g. the 50th percentile is the point below which half the observations lie. The percentiles that divide the data into four quarters are called: 
\begin{itemize}
\item[Q1] - 25th percentile or lower quartile
\item[Q2] - 50th percentile or median
\item[Q3] - 75th percentile or upper quartile
\end{itemize}

A measure of variability i.e. how the spread out the data are  is the difference between the point below which 25% of your data lie and the point below which 75% of your data lie i.e. Q3  - Q1. This is called the Interquartile Range  and it is commonly used for skewed data.
A graphical representation of the quartiles is called a Box plot 
% (Figure 1.3). 

It displays 
\begin{itemize}
\item[(a)] lower quartile 
\item[(b)] median 
\item[(c)] upper quartile  
\item[(d)] interquartile range (IQR)  
\item[(e)] whiskers of length = 1.5 IQR   
\item[(f)] outlying observations
\end{itemize}

%--------------------------------------------------------------%

