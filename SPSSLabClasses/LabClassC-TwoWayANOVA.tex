
\documentclass[]{article}
\voffset=-1.5cm
\oddsidemargin=0.0cm
\textwidth = 470pt

% http://www.strath.ac.uk/aer/materials/5furtherquantitativeresearchdesignandanalysis/unit6/whatislogisticregression/

% http://www.medcalc.org/manual/logistic_regression.php


\usepackage{amsmath}
\usepackage{graphicx}
\usepackage{amssymb}
\usepackage{framed}

\subsection{Introduction}
The two-way ANOVA compares the mean differences between groups that have been split on two independent variables (called factors). The primary purpose of a two-way ANOVA is to understand if there is an interaction between the two independent variables on the dependent variable. For example, you could use a two-way ANOVA to understand whether there is an interaction between gender and educational level on test anxiety amongst university students, where gender (males/females) and education level (undergraduate/postgraduate) are your independent variables, and test anxiety is your dependent variable. 

Alternately, you may want to determine whether there is an interaction between physical activity level and gender on blood cholesterol concentration in children, where physical activity (low/moderate/high) and gender (male/female) are your independent variables, and cholesterol concentration is your dependent variable.

The interaction term in a two-way ANOVA informs you whether the effect of one of your independent variables on the dependent variable is the same for all values of your other independent variable (and vice versa). For example, is the effect of gender (male/female) on test anxiety influenced by educational level (undergraduate/postgraduate)? Additionally, if a statistically significant interaction is found, you need to determine whether there are any "simple main effects", and if there are, what these effects are (we discuss this later in our guide).

Note: If you have three independent variables rather than two, you need a three-way ANOVA.

In this "quick start" guide, we show you how to carry out a two-way ANOVA using SPSS Statistics, as well as interpret and report the results from this test. However, before we introduce you to this procedure, you need to understand the different assumptions that your data must meet in order for a two-way ANOVA to give you a valid result. We discuss these assumptions next.

%==============================================================================%
\subsection{Assumptions}
When you choose to analyse your data using a two-way ANOVA, part of the process involves checking to make sure that the data you want to analyse can actually be analysed using a two-way ANOVA. You need to do this because it is only appropriate to use a two-way ANOVA if your data "passes" six assumptions that are required for a two-way ANOVA to give you a valid result. In practice, checking for these six assumptions means that you have a few more procedures to run through in SPSS Statistics when performing your analysis, as well as spend a little bit more time thinking about your data, but it is not a difficult task.

Before we introduce you to these six assumptions, do not be surprised if, when analysing your own data using SPSS Statistics, one or more of these assumptions is violated (i.e., is not met). This is not uncommon when working with real-world data rather than textbook examples, which often only show you how to carry out a two-way ANOVA when everything goes well! However, don’t worry. Even when your data fails certain assumptions, there is often a solution to overcome this. First, let’s take a look at these six assumptions:

\begin{description}
\item[Assumption 1:] Your dependent variable should be measured at the continuous level (i.e., they are interval or ratio variables). Examples of continuous variables include revision time (measured in hours), intelligence (measured using IQ score), exam performance (measured from 0 to 100), weight (measured in kg), and so forth. You can learn more about interval and ratio variables in our article: Types of Variable.
\item[Assumption 2:] Your two independent variables should each consist of two or more categorical, independent groups. Example independent variables that meet this criterion include gender (2 groups: male or female), ethnicity (3 groups: Caucasian, African American and Hispanic), profession (5 groups: surgeon, doctor, nurse, dentist, therapist), and so forth.
\item[Assumption 3:] You should have independence of observations, which means that there is no relationship between the observations in each group or between the groups themselves. For example, there must be different participants in each group with no participant being in more than one group. This is more of a study design issue than something you would test for, but it is an important assumption of the two-way ANOVA. If your study fails this assumption, you will need to use another statistical test instead of the two-way ANOVA (e.g., a repeated measures design). If you are unsure whether your study meets this assumption, you can use our Statistical Test Selector, which is part of our enhanced guides.
\item[Assumption 4:] There should be no significant outliers. Outliers are data points within your data that do not follow the usual pattern (e.g., in a study of 100 students' IQ scores, where the mean score was 108 with only a small variation between students, one student had a score of 156, which is very unusual, and may even put her in the top 1\% of IQ scores globally). The problem with outliers is that they can have a negative effect on the two-way ANOVA, reducing the accuracy of your results. Fortunately, when using SPSS Statistics to run a two-way ANOVA on your data, you can easily detect possible outliers. In our enhanced two-way ANOVA guide, we: (a) show you how to detect outliers using SPSS Statistics; and (b) discuss some of the options you have in order to deal with outliers.
\item[Assumption 5:] Your dependent variable should be approximately normally distributed for each combination of the groups of the two independent variables. Whilst this sounds a little tricky, it is easily tested for using SPSS Statistics. Also, when we talk about the two-way ANOVA only requiring approximately normal data, this is because it is quite "robust" to violations of normality, meaning the assumption can be a little violated and still provide valid results. You can test for normality using the Shapiro-Wilk test for normality, which is easily tested for using SPSS Statistics. In addition to showing you how to do this in our enhanced two-way ANOVA guide, we also explain what you can do if your data fails this assumption (i.e., if it fails it more than a little bit).
\item[Assumption 6:] There needs to be homogeneity of variances for each combination of the groups of the two independent variables. Again, whilst this sounds a little tricky, you can easily test this assumption in SPSS Statistics using Levene’s test for homogeneity of variances. In our enhanced two-way ANOVA guide, we (a) show you how to perform Levene’s test for homogeneity of variances in SPSS Statistics, (b) explain some of the things you will need to consider when interpreting your data, and (c) present possible ways to continue with your analysis if your data fails to meet this assumption.

\end{description}
You can check assumptions #4, #5 and #6 using SPSS Statistics. Before doing this, you should make sure that your data meets assumptions #1, #2 and #3, although you don’t need SPSS Statistics to do this. Just remember that if you do not run the statistical tests on these assumptions correctly, the results you get when running a two-way ANOVA might not be valid. This is why we dedicate a number of sections of our enhanced two-way ANOVA guide to help you get this right. You can find out about our enhanced content as a whole here, or more specifically, learn how we help with testing assumptions here.

In the section, Test Procedure in SPSS Statistics, we illustrate the SPSS Statistics procedure to perform a two-way ANOVA assuming that no assumptions have been violated. First, we set out the example we use to explain the two-way ANOVA procedure in SPSS Statistics.

%=====================================================================================%
Example
A researcher was interested in whether an individual's interest in politics was influenced by their level of education and gender. They recruited a random sample of participants to their study and asked them about their interest in politics, which they scored from 0 to 100, with higher scores indicating a greater interest in politics. The researcher then divided the participants by gender (Male/Female) and then again by level of education (School/College/University). Therefore, the dependent variable was "interest in politics", and the two independent variables were "gender" and "education".


%=====================================================================================%
\subsection{Setup in SPSS Statistics}
In SPSS Statistics, we separated the individuals into their appropriate groups by using two columns representing the two independent variables, and labelled them Gender and Edu_Level. For Gender, we coded "males" as 1 and "females" as 2, and for Edu_Level, we coded "school" as 1, "college" as 2 and "university" as 3. The participants' interest in politics – the dependent variable – was entered under the variable name, Int_Politics. The setup for this example can be seen below:

Data setup for a two-way ANOVA in SPSS
%=======================================================================%
If you are still unsure how to correctly set up your data in SPSS Statistics to carry out a two-way ANOVA, we show you all the required steps in our enhanced two-way ANOVA guide.


%=======================================================================%
\subsection{Test Procedure in SPSS Statistics}
The 14 steps below show you how to analyse your data using a two-way ANOVA in SPSS Statistics when the six assumptions in the previous section, Assumptions, have not been violated. At the end of these 14 steps, we show you how to interpret the results from this test. If you are looking for help to make sure your data meets assumptions #4, #5 and #6, which are required when using a two-way ANOVA and can be tested using SPSS Statistics, you can learn more in our enhanced guides here.

Click Analyze > General Linear Model > Univariate... on the top menu, as shown below:

Two-way ANOVA Menu

%=======================================================================%
You will be presented with the Univariate dialogue box, as shown below:

Two-way ANOVA Dialogue Box
Published with written permission from SPSS Statistics, IBM Corporation.
Transfer the dependent variable, Int_Politics, into the Dependent Variable: box, and transfer both independent variables, Gender and Edu_Level, into the Fixed Factor(s): box. You can do this by drag-and-dropping the variables into the respective boxes or by using the SPSS Right Arrow Button button. If you are using older versions of SPSS Statistics you will need to use the latter method. You will end up with a screen similar to that shown below:

Two-way ANOVA Dialogue Box

%=======================================================================%
Note: For this analysis, you will not need to worry about the Random Factor(s):, Covariate(s): or WLS Weight: boxes.

Click on the SPSS Plots Button button. You will be presented with the Univariate: Profile Plots dialogue box, as shown below:

Two-way ANOVA Plots Dialogue Box
Published with written permission from SPSS Statistics, IBM Corporation.
Transfer the independent variable, Edu_Level, from the Factors: box into the Horizontal Axis: box, and transfer the other independent variable, Gender, into the Separate Lines: box. You will be presented with the following screen:

Two-way ANOVA Plots Dialogue Box
Published with written permission from SPSS Statistics, IBM Corporation.
Note: It can help to put the independent variable with the greater number of groups in the Horizontal Axis: box.

Click the SPSS Add Button button. You will see that "Edu_Level*Gender" has been added to the Plots: box, as shown below:

Two-way ANOVA Plots Dialogue Box
Published with written permission from SPSS Statistics, IBM Corporation.
Click the  button. This will return you to the Univariate dialogue box.

Join the 10,000s of students, academics and professionals who rely on Laerd Statistics.TAKE THE TOUR PLANS & PRICING
Click the SPSS post-hoc Button button. You will be presented with the Univariate: Post Hoc Multiple Comparisons for Observed Means dialogue box, as shown below:

Two-way ANOVA post-hoc Dialogue Box
Published with written permission from SPSS Statistics, IBM Corporation.
Transfer Edu_Level from the Factor(s): box to the Post Hoc Tests for: box. This will make the –Equal Variances Assumed– area become active (lose the "grey sheen") and present you with some choices for which post hoc test to use. For this example, we are going to select Tukey, which is a good, all-round post hoc test.

Note: You only need to transfer independent variables that have more than two groups into the Post Hoc Tests for: box. This is why we do not transfer Gender.

You will finish up with the following screen:

Two-way ANOVA post-hoc Dialogue Box
Published with written permission from SPSS Statistics, IBM Corporation.
Click the  button to return to the Univariate dialogue box.

Click the SPSS Options Button button. This will present you with the Univariate: Options dialogue box, as shown below:

Two-way ANOVA Options Dialogue Box

%=======================================================================%
Transfer Gender, Edu\_Level and Gender*Edu_Level from the Factor(s) and Factor Interactions: box into the Display Means for: box. In the –Display– area, tick the Descriptive Statistics option. You will presented with the following screen:

Two-way ANOVA Options Dialogue Box
Published with written permission from SPSS Statistics, IBM Corporation.
Click the  button to return to the Univariate dialogue box.

Click the  button to generate the output.

Go to the next page for the SPSS Statistics output, discussion of simple main effects and an explanation of the output.


%=======================================================================%

\subsection{SPSS Statistics Output of the Two-way ANOVA}
SPSS Statistics generates quite a few tables in its output from a two-way ANOVA. In this section, we show you the main tables required to understand your results from the two-way ANOVA, including descriptives, between-subjects effects, Tukey post hoc tests (multiple comparisons), a plot of the results, and how to write up these results.

For a complete explanation of the output you have to interpret when checking your data for the six assumptions required to carry out a two-way ANOVA, see our enhanced guide. This includes relevant boxplots, and output from your Shapiro-Wilk test for normality and test for homogeneity of variances.

Finally, if you have a statistically significant interaction, you will also need to report simple main effects. Alternately, if you do not have a statistically significant interaction, there are other procedures you will have to follow. We show you these procedures in SPSS Statistics, as well as how to interpret and write up your results in our enhanced two-way ANOVA guide.

Below, we take you through each of the main tables required to understand your results from the two-way ANOVA.

Join the 10,000s of students, academics and professionals who rely on Laerd Statistics.TAKE THE TOUR PLANS & PRICING

%=======================================================================%
\subsection{Descriptive statistics}
You can find appropriate descriptive statistics for when you report the results of your two-way ANOVA in the aptly named "Descriptive Statistics" table, as shown below:

Output of two-way ANOVA in SPSS
Published with written permission from SPSS Statistics, IBM Corporation.
This table is very useful because it provides the mean and standard deviation for each combination of the groups of the independent variables (what is sometimes referred to as each "cell" of the design). In addition, the table provides "Total" rows, which allows means and standard deviations for groups only split by one independent variable, or none at all, to be known. This might be more useful if you do not have a statistically significant interaction.

SPSS Statisticstop ^
Plot of the results
The plot of the mean "interest in politics" score for each combination of groups of "Gender" and "Edu_level" are plotted in a line graph, as shown below:

Plot of the Results in two-way ANOVA in SPSS
Published with written permission from SPSS Statistics, IBM Corporation.
Although this graph is probably not of sufficient quality to present in your reports (you can edit its look in SPSS Statistics), it does tend to provide a good graphical illustration of your results. An interaction effect can usually be seen as a set of non-parallel lines. You can see from this graph that the lines do not appear to be parallel (with the lines actually crossing). You might expect there to be a statistically significant interaction, which we can confirm in the next section.

%=======================================================================%
\subsection{Statistical significance of the two-way ANOVA}
The actual result of the two-way ANOVA – namely, whether either of the two independent variables or their interaction are statistically significant – is shown in the Tests of Between-Subjects Effects table, as shown below:

Tests of Between-Subjects Effects Table in two-way ANOVA in SPSS
Published with written permission from SPSS Statistics, IBM Corporation.

The particular rows we are interested in are the "Gender", "Edu\_Level" and "Gender*Edu\_Level" rows, and these are highlighted above. These rows inform us whether our independent variables (the "Gender" and "Edu\_Level" rows) and their interaction (the "Gender*Edu\_Level" row) have a statistically significant effect on the dependent variable, "interest in politics". It is important to first look at the "Gender*Edu\_Level" interaction as this will determine how you can interpret your results (see our enhanced guide for more information). You can see from the "Sig." column that we have a statistically significant interaction at the p = .014 level. You may also wish to report the results of "Gender" and "Edu\_Level", but again, these need to be interpreted in the context of the interaction result. We can see from the table above that there was no statistically significant difference in mean interest in politics between males and females (p = .207), but there were statistically significant differences between educational levels (p < .0005).

%=======================================================================%
Post hoc tests – simple main effects in SPSS Statistics
When you have a statistically significant interaction, reporting the main effects can be misleading. Therefore, you will need to report the simple main effects. In our example, this would involve determining the mean difference in interest in politics between genders at each educational level, as well as between educational level for each gender. Unfortunately, SPSS Statistics does not allow you to do this using the graphical interface you will be familiar with, but requires you to use syntax. Therefore, in our enhanced two-way ANOVA guide, we show you the procedure for doing this in SPSS Statistics, as well as explaining how to interpret and write up the output from your simple main effects.

When you do not have a statistically significant interaction, we explain two options you have, as well as a procedure you can use in SPSS Statistics to deal with this issue.

%===================================================%
Multiple Comparisons Table
If you do not have a statistically significant interaction, you might interpret the Tukey post hoc test results for the different levels of education, which can be found in the Multiple Comparisons table, as shown below:

Tests of Between-Subjects Effects Table in two-way ANOVA in SPSS
Published with written permission from SPSS Statistics, IBM Corporation.
You can see from the table above that there is some repetition of the results, but regardless of which row we choose to read from, we are interested in the differences between (1) School and College, (2) School and University, and (3) College and University. From the results, we can see that there is a statistically significant difference between all three different educational levels (p < .0005).

%===================================================================================%
\subsection{Reporting the results of a two-way ANOVA}
You should emphasize the results from the interaction first before you mention the main effects. For example, you might report the result as:

General
A two-way ANOVA was conducted that examined the effect of gender and education level on interest in politics. There was a statistically significant interaction between the effects of gender and education level on interest in politics, F (2, 54) = 4.643, p = .014.

If you had a statistically significant interaction term and carried out the procedure for simple main effects in SPSS Statistics, you would also report these results. Briefly, you might report these as:

General
Simple main effects analysis showed that males were significantly more interested in politics than females when educated to university level (p = .002), but there were no differences between gender when educated to school (p = .465) or college level (p = .793).

In our enhanced two-way ANOVA guide, we show you how to write up the results from your assumptions tests and two-way ANOVA procedure, including simple main effects, if you need to report this in a dissertation/thesis, assignment or research report. We do this using the Harvard and APA styles. You can learn more about our enhanced content here.

\end{document}
