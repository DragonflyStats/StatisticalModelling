ANCOVA in SPSS Statistics

Introduction
The ANCOVA (analysis of covariance) can be thought of as an extension of the one-way ANOVA to incorporate a "covariate". Like the one-way ANOVA, the ANCOVA is used to determine whether there are any significant differences between the means of two or more independent (unrelated) groups (specifically, the adjusted means). However, the ANCOVA has the additional benefit of allowing you to "statistically control" for a third variable (sometimes known as a "confounding variable"), which may be negatively affecting your results. This third variable that could be confounding your results is the "covariate" that you include in an ANCOVA. If you are familiar with the ANCOVA, skip to the section of Assumptions. Alternately, the example below should help provide some clarity.

Imagine that you want to know if a hypnotherapy programme or wearing a nicotine patch could help smokers reduce the number of cigarettes they consumed per day. You recruit 150 smokers to take part in your experiment, who are randomly assigned to one of three groups: 50 in the "control group", 50 in the "hypnotherapy group" (treatment group A) and 50 in the "nicotine group" (treatment group B). The control group carry on smoking as usual, whilst "treatment group A" underwent the hypnotherapy programme and "treatment group B" wore the nicotine patches. You first measure the cigarette consumption of the smokers in each of the three groups at the start of the experiment, and again after the interventions (i.e., the hypnotherapy programme and nicotine programme) have finished. You could then use a one-way ANOVA with post hoc tests to (a) understand whether there were any statistically significant differences in cigarette consumption before and after the interventions amongst the three groups; and (b) if there were statistically significant differences, where these differences are (e.g., did cigarette consumption decrease only in the nicotine group, and by how many cigarettes per day). Let's imagine that after analysing this data using a one-way ANOVA, we found that the cigarette consumption of smokers in the nicotine group (treatment group B) dropped by 3 cigarettes per day compared to the control group, but just 2 cigarettes per day in the hypnotherapy group (treatment group A), and that this decrease was found to be statistically significant. With these results, we could conclude that both interventions were successful in reducing cigarette consumption. However, there is another possible explanation for our results; that a third, "confounding variable" explained the differences between our three groups, and not the interventions (i.e., the hypnotherapy or nicotine programmes). Let's imagine that this third variable could have been the difference in the number of cigarettes consumed by the 150 smokers at the start of the experiment (e.g., participants in all groups smoked a different number of cigarettes at the start of the experiment). By "randomly assigning" the 150 smokers to one of the three groups at the start of our experiment, we tried to minimize such sampling bias, but it is inevitable (given this study design) that participants would be different within the groups (i.e., differences in the number of cigarettes the smokers consumed in each of the three groups). Therefore, by making the initial cigarette consumption of the three groups a "covariate", the ANCOVA allows us to "statistically control" for the effect of these initial differences so that we can get a more accurate picture of the effect of the two treatments.

This "quick start" guide shows you how to carry out an ANCOVA using SPSS Statistics, as well as interpret and report the results from this test. Since the ANCOVA is often followed up with post hoc tests, we also show you how to carry these out using SPSS Statistics. However, before we introduce you to this procedure, you need to understand the different assumptions that your data must meet in order for an ANCOVA to give you a valid result. We discuss these assumptions next.

\subsection{Assumptions}
\begin{itemize}
\item 
When you choose to analyse your data using an ANCOVA, part of the process involves checking to make sure that the data you want to analyse can actually be analysed using an ANCOVA. You need to do this because it is only appropriate to use an ANCOVA if your data "passes" nine assumptions that are required for an ANCOVA to give you a valid result. In practice, checking for these nine assumptions just adds a little bit more time to your analysis, requiring you to click a few more buttons in SPSS Statistics when performing your analysis, as well as think a little bit more about your data, but it is not a difficult task.
\item 
Before we introduce you to these nine assumptions, do not be surprised if, when analysing your own data using SPSS Statistics, one or more of these assumptions is violated (i.e., is not met). This is not uncommon when working with real-world data rather than textbook examples, which often only show you how to carry out an ANCOVA when everything goes well! However, don’t worry. Even when your data fails certain assumptions, there is often a solution to overcome this. First, let’s take a look at these nine assumptions:
\end{itemize}


%---------------------------------------------------------%
Assumption #1: Your dependent variable and covariate variable should be measured on a continuous scale (i.e., they are measured at the interval or ratio level). Examples of variables that meet this criterion include revision time (measured in hours), intelligence (measured using IQ score), exam performance (measured from 0 to 100), weight (measured in kg), and so forth. You can learn more about continuous variables in our article: Types of Variable.
Assumption #2: Your independent variable should consist of two or more categorical, independent groups. Example independent variables that meet this criterion include gender (e.g., 2 groups: male and female), ethnicity (e.g., 3 groups: Caucasian, African American and Hispanic), physical activity level (e.g., 4 groups: sedentary, low, moderate and high), profession (e.g., 5 groups: surgeon, doctor, nurse, dentist, therapist), and so forth.
Assumption #3: You should have independence of observations, which means that there is no relationship between the observations in each group or between the groups themselves. For example, there must be different participants in each group with no participant being in more than one group. This is more of a study design issue than something you can test for, but it is an important assumption of an ANCOVA. If your study fails this assumption, you will need to use another statistical test instead of an ANCOVA (e.g., a repeated measures design). If you are unsure whether your study meets this assumption, you can use our Statistical Test Selector, which is part of our enhanced guides.
Assumption #4: There should be no significant outliers. Outliers are simply single data points within your data that do not follow the usual pattern (e.g., in a study of 100 students' IQ scores, where the mean score was 108 with only a small variation between students, one student had a score of 156, which is very unusual, and may even put her in the top 1% of IQ scores globally). The problem with outliers is that they can have a negative effect on the one-way ANCOVA, reducing the accuracy of your results. Fortunately, when using SPSS Statistics to run an ANCOVA on your data, you can easily detect possible outliers. In our enhanced ANCOVA guide, we: (a) show you how to detect outliers using SPSS Statistics; and (b) discuss some of the options you have in order to deal with outliers. You can learn more about our enhanced content here.
Assumption #5: Your dependent variable should be approximately normally distributed for each category of the independent variable. We talk about the ANCOVA only requiring approximately normal data because it is quite "robust" to violations of normality, meaning that assumption can be a little violated and still provide valid results. You can test for normality using two Shapiro-Wilk tests of normality: one to test the within-group residuals and one to test the overall model fit. Both of these are easily tested for using SPSS Statistics. In addition to showing you how to carry out these tests in our enhanced ANCOVA guide, we also explain what you can do if your data fails this assumption (i.e., if it fails it more than a little bit).
Assumption #6: There needs to be homogeneity of variances. You can test this assumption in SPSS Statistics using Levene's test for homogeneity of variances. In our enhanced ANCOVA guide, we (a) show you how to perform Levene’s test for homogeneity of variances in SPSS Statistics, (b) explain some of the things you will need to consider when interpreting your data, and (c) present possible ways to continue with your analysis if your data fails to meet this assumption.
Assumption #7: The covariate is linearly related to the dependent variable at each level of the independent variable. You can test this assumption in SPSS Statistics by plotting a grouped scatterplot of the covariate, post-test scores of the dependent variable and independent variable. In our enhanced ANCOVA guide, we show you how to (a) produce this grouped scatterplot in SPSS Statistics, (b) interpret the grouped scatterplot, and (c) present possible ways to continue with your analysis if your data fails to meet this assumption.
Assumption #8: There needs to be homoscedasticity. You can test this assumption in SPSS Statistics by plotting a scatterplot of the standardized residuals against the predicted values. In our enhanced ANCOVA guide, we (a) show you how to produce a scatterplot in SPSS Statistics to test for homoscedasticity, (b) explain some of the things you will need to consider when interpreting your data, and (c) present possible ways to continue with your analysis if your data fails to meet this assumption.
Assumption #9: There needs to be homogeneity of regression slopes, which means that there is no interaction between the covariate and the independent variable. By default, SPSS Statistics does not include an interaction term between a covariate and a factor in its GLM procedure so that you can test this. Therefore, in our enhanced ANCOVA guide, we (a) show you how to test for homogeneity of regression slopes separately from the main ANCOVA procedure using SPSS Statistics, (b) interpret the output SPSS Statistics produces, and (c) explain the implications of meeting or violating this assumption.
You can check assumptions #4, #5, #6, #7, #8 and #9 using SPSS Statistics. Before doing this, you should make sure that your data meets assumptions #1, #2 and #3, although you don't need SPSS Statistics to do this. When moving on to assumptions #4, #5, #6, #7, #8 and #9, we suggest testing them in this order because it represents an order where, if a violation to the assumption is not correctable, you will no longer be able to use an ANCOVA (although you may be able to run another statistical test on your data instead). Just remember that if you do not run the statistical tests on these assumptions correctly, the results you get when running an ANCOVA might not be valid. This is why we dedicate a number of sections of our enhanced ANCOVA guide to help you get this right. You can find out about our enhanced content as a whole here, or more specifically, learn how we help with testing assumptions here.

In the section, Test Procedure in SPSS Statistics, we illustrate the SPSS Statistics procedure to perform an ANCOVA, assuming that no assumptions have been violated. First, we set out the example we use to explain the ANCOVA procedure in SPSS Statistics.

TAKE THE TOUR

PLANS & PRICING
SPSS Statisticstop ^
Example
A researcher was interested in determining whether a six-week low- or high-intensity exercise-training programme was best at reducing blood cholesterol concentrations in middle-aged men. Both exercise programmes were designed so that the same number of calories was expended in the low- and high-intensity groups. As such, the duration of exercise differed between groups. The researcher expected that any reduction in cholesterol concentration elicited by the interventions would also depend on the participant's initial cholesterol concentration. As such, the researcher wanted to use pre-intervention cholesterol concentration as a covariate when comparing the post-intervention cholesterol concentrations between the interventions and a control group. Therefore, the researcher ran an ANCOVA with: (a) post-intervention cholesterol concentration (post) as the dependent variable; (b) the control and two intervention groups as levels of the independent variable, group; and (c) the pre-intervention cholesterol concentrations as the covariate, pre.

SPSS Statisticstop ^
Setup in SPSS Statistics
In SPSS Statistics, we entered three variables: (1) the dependent variable, post, which is the post-intervention cholesterol concentration; (2) the independent variable, group, which has three categories: "control", "Int_1" (representing the low-intensity exercise intervention), and "Int_2" (representing the high-intensity exercise intervention); and (3) pre, which represents the pre-intervention cholesterol concentrations. In our enhanced ANCOVA guide, we show you how to correctly enter data in SPSS Statistics to run an ANCOVA. You can learn about our enhanced data setup content in general here. Alternately, we have a generic, "quick start" guide to show you how to enter data into SPSS Statistics, available here.

SPSS Statisticstop ^
Test Procedure in SPSS Statistics
The nine steps below show you how to analyse your data using an ANCOVA in SPSS Statistics when the nine assumptions in the previous section, Assumptions, have not been violated. At the end of these nine steps, we show you how to interpret the results from this test. If you are looking for help to make sure your data meets assumptions #4, #5, #6, #7, #8 and #9, which are required when using an ANCOVA and can be tested using SPSS Statistics, you can learn more about our enhanced content here.

Click Analyze > General Linear Model > Univariate... on the main menu, as shown below:

ANCOVA Menu
Published with written permission from SPSS Statistics, IBM Corporation.
You will be presented with the Univariate dialogue box, as shown below:

ANCOVA Univariate dialogue box
Published with written permission from SPSS Statistics, IBM Corporation.
Your previous actions should still be highlighted. If not, transfer the variables as above.

Click the  button and you will be presented with the Univariate: Model dialogue box, as shown below:

ANCOVA Univariate Model dialogue box
Published with written permission from SPSS Statistics, IBM Corporation.
Note: If you have not run the assumptions to this test, you will be presented with a different screen to the one above – one without variables in the Model: box.

Click Full Factorial in the –Specify Model– area. This will deactivate the Factors & covariates: and Model: boxes and the options therein, as shown below:

ANCOVA Univariate Model dialogue box checked
Published with written permission from SPSS Statistics, IBM Corporation.
Note: Now that you have tested the assumption of homogeneity of regression slopes by including the interaction term, it is no longer needed. By selecting Full Factorial you are removing the interaction term from the model as SPSS Statistics does not keep covariate interaction terms in the model by default.

Click the  button and you will be returned to the Univariate dialogue box.

Click the  button. You will be presented with the Univariate: Options dialogue box, as shown below:

ANCOVA Univariate Options dialogue box
Published with written permission from SPSS Statistics, IBM Corporation.
Transfer the variable, group, from the Factor(s) and Factor Interactions: box to the Display Means for: box using the  button. Then, check Compare main effects, which will activate the Confidence interval adjustment: option. From this drop-down menu, select "Bonferroni" from the options. Also, select Descriptive statistics and Estimates of effect size in the –Display– area, as shown below:


Published with written permission from SPSS Statistics, IBM Corporation.
Click the  button and you will be returned to the Univariate dialogue box.

Click the  button. This will generate your output.
