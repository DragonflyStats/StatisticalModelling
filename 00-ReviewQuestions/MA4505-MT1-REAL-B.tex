\documentclass[a4paper,12pt]{article}
%%%%%%%%%%%%%%%%%%%%%%%%%%%%%%%%%%%%%%%%%%%%%%%%%%%%%%%%%%%%%%%%%%%%%%%%%%%%%%%%%%%%%%%%%%%%%%%%%%%%%%%%%%%%%%%%%%%%%%%%%%%%%%%%%%%%%%%%%%%%%%%%%%%%%%%%%%%%%%%%%%%%%%%%%%%%%%%%%%%%%%%%%%%%%%%%%%%%%%%%%%%%%%%%%%%%%%%%%%%%%%%%%%%%%%%%%%%%%%%%%%%%%%%%%%%%
\usepackage{eurosym}
\usepackage{vmargin}
\usepackage{amsmath}
\usepackage{graphics}
\usepackage{framed}
\usepackage{epsfig}
\usepackage{subfigure}
\usepackage{fancyhdr}

\setcounter{MaxMatrixCols}{10}
%TCIDATA{OutputFilter=LATEX.DLL}
%TCIDATA{Version=5.00.0.2570}
%TCIDATA{<META NAME="SaveForMode"CONTENT="1">}
%TCIDATA{LastRevised=Wednesday, February 23, 201113:24:34}
%TCIDATA{<META NAME="GraphicsSave" CONTENT="32">}
%TCIDATA{Language=American English}

\pagestyle{fancy}
\setmarginsrb{20mm}{0mm}{20mm}{25mm}{12mm}{11mm}{0mm}{11mm}
\lhead{MA4505} \rhead{Kevin O'Brien} \chead{Midterm
Assessment Paper 1 } %\input{tcilatex}

\begin{document}
\begin{center}
	\includegraphics[scale=0.60]{images/shieldtransparent2}
\end{center}

\begin{center}
	\vspace{1cm}
	\large \bf {FACULTY OF SCIENCE AND ENGINEERING} \\[0.5cm]
	\normalsize DEPARTMENT OF MATHEMATICS AND STATISTICS \\[1.25cm]
	\large \bf {MID-TERM ASSESSMENT EXAMINATION 1} \\[1.0cm]
\end{center}

\begin{tabular}{ll}
%	MODULE CODE: MA4505 & SEMESTER: Autumn 2015\\[1cm]
	MODULE TITLE: Applied Statistic for Administration  & DURATION OF EXAM: 45 minutes \\[1cm]
	LECTURER: Mr. Kevin O'Brien & GRADING SCHEME: 15 marks \\
%	& \phantom{GRADING SCHEME:} \footnotesize {15\% of total module marks} \\[0.2cm]
%	\\[1cm]
\end{tabular}
\bigskip
\begin{center}
	{\bf INSTRUCTIONS TO CANDIDATES}
\end{center}

\begin{itemize} 
	\item This exam will start at 12:05, and will last 45 minutes.
	
	\item Each question will be worth either 1 or 2 Marks. There are 15 Marks worth of questions.
	\item All questions must be attempted (LENS students please see below)
	
	\item Write all of your answers in the exam script. Write the script number on any other documents you submit.
	
	\item It is your responsibility to return the script to collection box. An audit of scripts will take place immediately after the exam. If your script is account for in that audit,  you are deemed to be absent, and will receive no marks.
	
	\item \textbf{IMPORTANT for LENS Student:}
	Specifically approved LENS students have to answer any selection of questions that have an aggregate mark of 12 Marks.  
	\begin{itemize}
		\item They may skip any three of the 1-Mark Questions
		\item OR - They may skip a 1-Mark Question and a 2-Mark Question
		\item The mark will be rescaled by 125 \%.
		\item They are advised to skip questions that are indicated by an asterisk symbol (``$\ast$"), but it is not compulsory that they do so.
	\end{itemize}
	
	
\end{itemize}
\newpage
\section*{Attempt ALL questions}

\bigskip
\subsection*{Q1. Dixon Q Test For Outliers (4 Marks)}

The typing speeds for one group of 12 Engineering students were recorded both at the beginning of year 1 of their studies. The results (in words per minute) are given below:

\begin{center}
	\begin{tabular}{|c|c|c|c|c|c|}
		\hline
		% Subject& A& B& C& D& E &F &G &H \\ \hline
		118 & 146 & 149 & 142 & 170& 153\\ \hline
		137 & 161 & 156& 165&  178& 159
		\\ \hline
	\end{tabular}
\end{center}
Use the Dixon Q-test to determine if the lowest value (118) is an outlier. You may assume a significance level of 5\%.
\begin{itemize}
	\item[i.](1 Mark)	State the Null and Alternative Hypothesis for this test.
	\item[ii.](1 Marks) Compute the test statistic
	\item[iii.](1 Mark) State the appropriate critical value.
	\item[iv.](1 Mark) What is your conclusion to this procedure
\end{itemize}

%Calculate a 95\% confidence interval for the difference between the mean number of marks obtained by males and females in the population of school leavers as a whole.
%(7 marks)

\newpage
%\newpage
\subsection*{Q2. Normal Distribution (3 Marks)} % Normal %6 MARKS
Assume that the diameter of a critical component is normally distributed with a Mean of 100mm and a Standard Deviation of 5mm. You are required  to estimate the approximate probability of the following measurements occurring on an individual component.
\begin{itemize}
	\item[i.](1 Mark)	Greater than 104.1mm
	\item[ii.](2 Marks) Less than 95.2 mm
%	\item [iii.](2 Marks)[$\ast$] Between 94.2 and 103 mm
\end{itemize}
\bigskip
\noindent Use the normal tables to determine the probabilities for the above exercises. You are required to show all of your workings.
\newpage
(Write Your Answers Here)
\newpage

\subsection*{Q3. Chi-Square Test (8 Marks)} %4 

A market research survey was carried out to assess preferences for three brands of chocolate bar, A, B, and C. 
The study group was categorised by gender to determine any difference in preferences.


{
	\large
\begin{center}
\begin{tabular}{|c|c|c|c|c|}
\hline
& A & B & C &  Total\\ \hline
Females & 50 & 70 & 80 & 200 \\ \hline
Males   & 90 & 50 & 20 &  160\\ \hline
Total & 140 & 120 & 100 & 360\\ \hline
\end{tabular} 
\end{center}
}
\begin{itemize}
	\item[i.](1 Mark)[$\ast$] Formally state the null and alternative hypotheses.
	\item[ii.] (2 Marks) Compute the cell values expected under the null hypothesis. Show your workings for two cells.
	\item[iii.](3 Marks) Compute the Test Statistic.
	\item[iv.](1 Mark)[$\ast$] State the appropriate Critical Value for this hypothesis test.
	\item[v.](1 Mark)[$\ast$] Discuss your conclusion to this test, supporting your statement with reference to appropriate values.
\end{itemize}
\newpage
(Write Your Answers Here)
%\end{itemize}
% -- Part 2 - Hypothesis Testing Computation
%
% 1 Mark
% 1 Mark
%Compute the pooled variance for the aggregate sample
% 1 Mark Standard Error
% 1 Mark Test Statistic
% 1 Mark Appropriate level of significance
% 1 Mark Appropriate degrees of freedom
% 1 Mark Appropriate Critical value
% 1 MArk Discuss your conclusion to this test, supporting your statement with reference to appropriate values.

\newpage


\section*{Formulae and Tables}
\subsection*{Critical Values for Dixon Q Test}
{
	\Large
	\begin{center}
		\begin{tabular}{|c|c|c|c|}
			\hline  N  & $\alpha=0.10$  & $\alpha=0.05$  & $\alpha=0.01$  \\ \hline
			3  & 0.941 & 0.970 & 0.994 \\ \hline
			4  & 0.765 & 0.829 & 0.926 \\ \hline
			5  & 0.642 & 0.710  & 0.821 \\ \hline
			6  & 0.560 & 0.625 & 0.740 \\ \hline
			7  & 0.507 & 0.568 & 0.680  \\ \hline
			8  & 0.468 & 0.526 & 0.634 \\ \hline
			9  & 0.437 & 0.493 & 0.598 \\ \hline
			10 & 0.412 & 0.466 & 0.568 \\ \hline
			11 & 0.392 & 0.444 & 0.542 \\ \hline
			12 & 0.376 & 0.426 & 0.522 \\ \hline
			13 & 0.361 & 0.410 & 0.503 \\ \hline
			14 & 0.349 & 0.396 & 0.488 \\ \hline
			15 & 0.338 & 0.384 & 0.475 \\ \hline
			16 & 0.329 & 0.374 & 0.463 \\ \hline
		\end{tabular} 
	\end{center}
}
\newpage
\subsection*{Critical Values for Chi Square Test}
{
	\Large
	\begin{center}
		\begin{tabular}{|c|c|c|c|c|}
			\hline 
			d.f.	&	$\alpha=0.10$	&	$\alpha=0.05$	&	$\alpha=0.01$	&	$\alpha=0.001$	\\ \hline
			1	& 	2.705	&	3.841	&	6.634	&	10.827	\\ \hline
			2	&	4.605	&	5.991	&	7.378	&	9.21	\\ \hline
			3	&	6.251	&	7.815	&	9.348	&	11.345	\\ \hline
			4	&	7.779	&	9.488	&	11.143	&	13.277	\\ \hline
			5	&	9.236	&	11.07	&	12.833	&	15.086	\\ \hline
			6	&	10.645	&	12.592	&	14.449	&	16.812	\\ \hline
			7	&	12.017	&	14.067	&	16.013	&	18.475	\\ \hline
			8	&	13.362	&	15.507	&	17.535	&	20.09	\\ \hline
			9	&	14.684	&	16.919	&	19.023	&	21.666	\\ \hline
			10	&	15.987	&	18.307	&	20.483	&	23.209	\\ \hline
		\end{tabular} 
	\end{center}
}
\bigskip
\subsection*{Test Statistic for Chi-Square Test}
{
\Large
\[ \chi^2_{TS} = \sum \frac{(\mbox{Observed} -\mbox{Expected} )^2}{\mbox{Expected} }\]
}

\end{document}
\subsection*{Confidence Intervals}
{\bf One sample}
\begin{eqnarray*} S.E.(\bar{X})&=&\frac{\sigma}{\sqrt{n}}.\\\\
S.E.(\hat{P})&=&\sqrt{\frac{\hat{p}\times(100-\hat{p})}{n}}.\\
\end{eqnarray*}
{\bf Two samples}
\begin{eqnarray*}
S.E.(\bar{X}_1-\bar{X}_2)&=&\sqrt{\frac{\sigma^2_1}{n_1}+\frac{\sigma_2^2}{n_2}}.\\\\
S.E.(\hat{P_1}-\hat{P_2})&=&\sqrt{\frac{\hat{p}_1\times(100-\hat{p}_1)}{n_1}+\frac{\hat{p}_2\times(100-\hat{p}_2)}{n_2}}.\\\\
\end{eqnarray*}
\subsection*{Hypothesis tests}
{\bf One sample}
\begin{eqnarray*}
S.E.(\bar{X})&=&\frac{\sigma}{\sqrt{n}}.\\\\
S.E.(\pi)&=&\sqrt{\frac{\pi\times(100-\pi)}{n}}
\end{eqnarray*}
{\bf Two large independent samples}
\begin{eqnarray*}
S.E.(\bar{X}_1-\bar{X}_2)&=&\sqrt{\frac{\sigma^2_1}{n_1}+\frac{\sigma_2^2}{n_2}}.\\\\
S.E.(\hat{P_1}-\hat{P_2})&=&\sqrt{\left(\bar{p}\times(100-\bar{p})\right)\left(\frac{1}{n_1}+\frac{1}{n_2}\right)}.\\
\end{eqnarray*}
{\bf Two small independent samples}
\begin{eqnarray*}
S.E.(\bar{X}_1-\bar{X}_2)&=&\sqrt{s_p^2\left(\frac{1}{n_1}+\frac{1}{n_2}\right)}.\\\\
s_p^2&=&\frac{s_1^2(n_1-1)+s_2^2(n_2-1)}{n_1+n_2-2}.\\
\end{eqnarray*}
{\bf Paired sample}
\begin{eqnarray*}
S.E.(\bar{d})&=&\frac{s_d}{\sqrt{n}}.\\\\
\end{eqnarray*}
{\bf Standard Deviation of case-wise differences (computational formula)}
\begin{eqnarray*}
s_d = \sqrt{ {\sum d_i^2 - n\bar{d}^2 \over n-1}}.\\\\
\end{eqnarray*}
\end{document}
% -- Part 3 - Confidence Interval

% 2 Marks Using previously calculated values, compute the confidence interval
