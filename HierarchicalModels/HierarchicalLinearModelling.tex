Hierarchical Linear Modelling

Multilevel models (also hierarchical linear models, nested models, mixed models, random coefficient, random-effects models, random parameter models, or split-plot designs) are statistical models of parameters that vary at more than one level.[1] These models can be seen as generalizations of linear models (in particular, linear regression), although they can also extend to non-linear models. These models became much more popular after sufficient computing power and software became available.[1]

Multilevel models are particularly appropriate for research designs where data for participants are organized at more than one level (i.e., nested data).[2] The units of analysis are usually individuals (at a lower level) who are nested within contextual/aggregate units (at a higher level).[3] While the lowest level of data in multilevel models is usually an individual, repeated measurements of individuals may also be examined.[2] As such, multilevel models provide an alternative type of analysis for univariate or multivariate analysis of repeated measures. Individual differences in growth curves may be examined (see growth model).[2] Furthermore, multilevel models can be used as an alternative to ANCOVA, where scores on the dependent variable are adjusted for covariates (i.e., individual differences) before testing treatment differences.[4] Multilevel models are able to analyze these experiments without the assumptions of homogeneity-of-regression slopes that is required by ANCOVA.[2]

Multilevel models can be used on data with many levels, although 2-level models are the most common and the rest of this article deals only with these. The dependent variable must be examined at the lowest level of analysis.[1]

%-------------------

Level 1 regression equation[edit]
When there is a single level 1 independent variable, the level 1 model is:


 Y_{ij} = \beta_{0j} + \beta_{1j} X_{ij} + e_{ij}



Y_{ij}  refers to the score on the dependent variable for an individual observation at Level 1 (subscript i refers to individual case, subscript j refers to the group).
X_{ij}   refers to the Level 1 predictor.
\beta_{0j}   refers to the intercept of the dependent variable in group j (Level 2).
 \beta_{1j} refers to the slope for the relationship in group j (Level 2) between the Level 1 predictor and the dependent variable.
 e_{ij} refers to the random errors of prediction for the Level 1 equation (it is also sometimes referred to as r_{ij}). At Level 1, both the intercepts and slopes in the groups can be either fixed (meaning that all groups have the same values, although in the real world this would be a rare occurrence), non-randomly varying (meaning that the intercepts and/or slopes are predictable from an independent variable at Level 2), or randomly varying (meaning that the intercepts and/or slopes are different in the different groups, and that each have their own overall mean and variance).[2]
When there are multiple level 1, the model can be expanded by substituting vectors and matrices in the equation.

Level 2 regression equation[edit]
The dependent variables are the intercepts and the slopes for the independent variables at Level 1 in the groups of Level 2.

\beta_{0j} = \gamma_{00} + \gamma_{01} W_j + u_{0j}

\beta_{1j} = \gamma_{10} + u_{1j}

\gamma_{00} refers to the overall intercept. This is the grand mean of the scores on the dependent variable across all the groups when all the predictors are equal to 0.
W_j refers to the Level 2 predictor.
\gamma_{01} refers to the overall regression coefficient, or the slope, between the dependent variable and the Level 2 predictor.
u_{0j} refers to the random error component for the deviation of the intercept of a group from the overall intercept.
\gamma_{10} refers to the overall regression coefficient, or the slope, between the dependent variable and the Level 1 predictor.
u_{1j} refers to the error component for the slope (meaning the deviation of the group slopes from the overall slope).[2]
Types of models[edit]
Before conducting a multilevel model analysis, a researcher must decide on several aspects, including which predictors are to be included in the analysis, if any. Second, the researcher must decide whether parameter values (i.e., the elements that will be estimated) will be fixed or random.[2][4] Fixed parameters are composed of a constant over all the groups, whereas a random parameter has a different value for each of the groups. Additionally, the researcher must decide whether to employ a maximum likelihood estimation or a restricted maximum likelihood estimation type.[2]

Random intercepts model[edit]
A random intercepts model is a model in which intercepts are allowed to vary, and therefore, the scores on the dependent variable for each individual observation are predicted by the intercept that varies across groups.[4][5] This model assumes that slopes are fixed (the same across different contexts). In addition, this model provides information about intraclass correlations, which are helpful in determining whether multilevel models are required in the first place.[2]

Random slopes model[edit]
A random slopes model is a model in which slopes are allowed to vary, and therefore, the slopes are different across groups. This model assumes that intercepts are fixed (the same across different contexts).[4]

Random intercepts and slopes model[edit]
A model that includes both random intercepts and random slopes is likely the most realistic type of model, although it is also the most complex. In this model, both intercepts and slopes are allowed to vary across groups, meaning that they are different in different contexts.[4]

Developing a multilevel model[edit]
In order to conduct a multilevel model analysis, one would start with fixed coefficients (slopes and intercepts). One aspect would be allowed to vary at a time (that is, would be changed), and compared with the previous model in order to assess better model fit.[1] There are three different questions that a researcher would ask in assessing a model. First, is it a good model? Second, is a more complex model better? Third, what contribution do individual predictors make to the model?

In order to assess models, different model fit statistics would be examined.[2] One such statistic is the chi-square likelihood-ratio test, which assesses the difference between models. The likelihood-ratio test can be employed for model building in general, for examining what happens when effects in a model are allowed to vary, and when testing a dummy-coded categorical variable as a single effect.[2] However, the test can only be used when models are nested (meaning that a more complex model includes all of the effects of a simpler model). When testing non-nested models, comparisons between models can be made using the Akaike information criterion (AIC) or the Bayesian information criterion (BIC), among others.[1][2][4] See further Model selection.

Assumptions[edit]
Multilevel models have the same assumptions as other major general linear models (e.g., ANOVA, regression), but some of the assumptions are modified for the hierarchical nature of the design (i.e., nested data).

Linearity[edit]
The assumption of linearity states that there is a rectilinear (straight-line, as opposed to non-linear or U-shaped) relationship between variables.[6] However, the model can be extended to nonlinear relationships. [7]

Normality[edit]
The assumption of normality states that the error terms at every level of the model are normally distributed.[6]

Homoscedasticity[edit]
The assumption of homoscedasticity, also known as homogeneity of variance, assumes equality of population variances.[6]

Independence of observations[edit]
Independence is an assumption of general linear models, which states that cases are random samples from the population and that scores on the dependent variable are independent of each other.[6] One of the main purposes of multilevel models is to deal with cases where the assumption of independence is violated; multilevel models do, however, assume that 1) the level 1 and level 2 residuals are uncorrelated and 2) The errors (as measured by the residuals) at the highest level are uncorrelated. [8]

Statistical tests[edit]
The type of statistical tests that are employed in multilevel models depend on whether one is examining fixed effects or variance components. When examining fixed effects, the tests are compared with the standard error of the fixed effect, which results in a Z-test.[4] A t-test can also be computed. When computing a t-test, it is important to keep in mind the degrees of freedom, which will depend on the level of the predictor (e.g., level 1 predictor or level 2 predictor).[4] For a level 1 predictor, the degrees of freedom are based on the number of level 1 predictors, the number of groups and the number of individual observations. For a level 2 predictor, the degrees of freedom are based on the number of level 2 predictors and the number of groups.[4]

Statistical power[edit]
Statistical power for multilevel models differs depending on whether it is level 1 or level 2 effects that are being examined. Power for level 1 effects is dependent upon the number of individual observations, whereas the power for level 2 effects is dependent upon the number of groups.[9] To conduct research with sufficient power, large sample sizes are required in multilevel models. However, the number of individual observations in groups is not as important as the number of groups in a study. In order to detect cross-level interactions, given that the group sizes are not too small, recommendations have been made that at least 20 groups are needed.[9] The issue of statistical power in multilevel models is complicated by the fact that power varies as a function of effect size and intraclass correlations, it differs for fixed effects versus random effects, and it changes depending on the number of groups and the number of individual observations per group.[9]

Applications of multilevel models[edit]
Level[edit]
The concept of level is the keystone of this approach. In an educational research example, the levels for a 2-level model might be:

pupil
class
However, if one were studying multiple schools and multiple school districts, a 4-level model could be:

pupil
class
school
district
The researcher must establish for each variable the level at which it was measured. In this example "test score" might be measured at pupil level, "teacher experience" at class level, "school funding" at school level, and "urban" at district level.

Example[edit]
As a simple example, consider a basic linear regression model that predicts income as a function of age, class, gender and race. It might then be observed that income levels also vary depending on the city and state of residence. A simple way to incorporate this into the regression model would be to add an additional independent categorical variable to account for the location (i.e. a set of additional binary predictors and associated regression coefficients, one per location). This would have the effect of shifting the mean income up or down — but it would still assume, for example, that the effect of race and gender on income is the same everywhere. In reality, this is unlikely to be the case — different local laws, different retirement policies, differences in level of racial prejudice, etc. are likely to cause all of the predictors to have different sorts of effects in different locales.

In other words, a simple linear regression model might, for example, predict that a given randomly sampled person in Seattle would have an average yearly income $10,000 higher than a similar person in Mobile, Alabama. However, it would also predict, for example, that a white person might have an average income $7,000 above a black person, and a 65-year-old might have an income $3,000 below a 45-year-old, in both cases regardless of location. A multilevel model, however, would allow for different regression coefficients for each predictor in each location. Essentially, it would assume that people in a given location have correlated incomes generated by a single set of regression coefficients, whereas people in another location have incomes generated by a different set of coefficients. Meanwhile, the coefficients themselves are assumed to be correlated and generated from a single set of hyperparameters. Additional levels are possible: For example, people might be grouped by cities, and the city-level regression coefficients grouped by state, and the state-level coefficients generated from a single hyper-hyperparameter.

Multilevel models are a subclass of hierarchical Bayesian models, which are general models with multiple levels of random variables and arbitrary relationships among the different variables. Multilevel analysis has been extended to include multilevel structural equation modeling, multilevel latent class modeling, and other more general models.

Uses of multilevel models[edit]
Multilevel models have been used in education research or geographical research, to estimate separately the variance between pupils within the same school, and the variance between schools. In psychological applications, the multiple levels are items in an instrument, individuals, and families. In sociological applications, multilevel models are used to examine individuals embedded within regions or countries. In organizational psychology research, data from individuals must often be nested within teams or other functional units.

Different covariables may be relevant on different levels. They can be used for longitudinal studies, as with growth studies, to separate changes within one individual and differences between individuals.

Cross-level interactions may also be of substantive interest; for example, when a slope is allowed to vary randomly, a level-2 predictor may be included in the slope formula for the level-1 covariate. For example, one may estimate the interaction of race and neighborhood so that an estimate of the interaction between an individual's characteristics and the context.
