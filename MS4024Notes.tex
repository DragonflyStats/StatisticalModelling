\documentclass[12pt, a4paper]{report}
\usepackage{epsfig}
\usepackage{subfigure}
%\usepackage{amscd}
\usepackage{amssymb}
\usepackage{graphicx}
%\usepackage{amscd}
\usepackage{amssymb}
\usepackage{amsthm, amsmath}
\usepackage{amsbsy}
\usepackage[usenames]{color}
%\usepackage{listings}
%%llstset{% general command to set parameter(s)
%basicstyle=\small, % print whole listing small
%keywordstyle=\color{red}\itshape,
%% underlined bold black keywords
%commentstyle=\color{blue}, % white comments
%stringstyle=\ttfamily, % typewriter type for strings
%showstringspaces=false,
%numbers=left, numberstyle=\tiny, stepnumber=1, numbersep=5pt, %
%frame=shadowbox,
%rulesepcolor=\color{black},
%,columns=fullflexible
%} %
%%\usepackage[dvips]{graphicx}
%\usepackage{natbib}
\bibliographystyle{chicago}
\usepackage{vmargin}
% left top textwidth textheight headheight
% headsep footheight footskip
\setmargins{3.0cm}{2.5cm}{15.5 cm}{22cm}{0.5cm}{0cm}{1cm}{1cm}
\renewcommand{\baselinestretch}{1.1}
\pagenumbering{arabic}
\theoremstyle{plain}
\newtheorem{theorem}{Theorem}[section]
\newtheorem{corollary}[theorem]{Corollary}
\newtheorem{ill}[theorem]{Example}
\newtheorem{lemma}[theorem]{Lemma}
\newtheorem{proposition}[theorem]{Proposition}
\newtheorem{conjecture}[theorem]{Conjecture}
\newtheorem{axiom}{Axiom}
\theoremstyle{definition}
\newtheorem{definition}{Definition}[section]
\newtheorem{notation}{Notation}
\theoremstyle{remark}
\newtheorem{remark}{Remark}[section]
\newtheorem{example}{Example}[section]
\renewcommand{\thenotation}{}
\renewcommand{\thetable}{\thesection.\arabic{table}}
\renewcommand{\thefigure}{\thesection.\arabic{figure}}
\title{Introduction to Computer Aided Data Analysis}
\author{ } \date{ }
\addcontentsline{toc}{section}{Bibliography}


\begin{document}
\author{Kevin O'Brien}
\title{Regression}


%\begin{figure}
%  % Requires \usepackage{graphicx}
%  \includegraphics[scale=1]{MTCARSmpgwt.png}\\
%  \caption{Scatterplot}\label{mpgwt}
%\end{figure}


\section{Regression}
The basic form of a formula is ``response $\sim$ model".

\section{Analysis of Variance}
Analysis of variance (ANOVA) is a collection of statistical models, and their associated procedures, in which the observed variance in a particular variable is partitioned into components attributable to different sources of variation.



\subsection{Multiple Linear Regression}
The basic model for multiple regression analysis is

\[
y = b_0 + b_1x_1 + \cdots  + b_kx_k + e
\]


\subsection{Example: MTCars}
The data was extracted from the 1974 Motor Trend US magazine, and comprises fuel consumption and 10 aspects of automobile design and performance for 32 automobiles (1973�74 models).

A data frame with 32 observations on 11 variables. Let
us assume that the variable mpg is the response variable, with the other ten being predictor variables.

\begin{itemize}
\item  \textbf{mpg}  Miles/(US) gallon
\item  \textbf{cyl}  Number of cylinders
\item  \textbf{disp}  Displacement (cu.in.)
\item  \textbf{hp}  Gross horsepower
\item  \textbf{drat}  Rear axle ratio
\item  \textbf{wt}  Weight (lb/1000)
\item  \textbf{qsec}  1/4 mile time
\item  \textbf{vs}  V/S
\item  \textbf{am}  Transmission (0 = automatic, 1 = manual)
\item  \textbf{gear}  Number of forward gears
\item  \textbf{carb}  Number of carburetors
\end{itemize}

\subsection{Model specification and output}
Specification of a multiple regression analysis is done by setting up a
model formula with + between the explanatory variables:
\begin{verbatim}
lm(mpg~cyl+disp+hp+drat+wt+qsec+vs+am+gear+carb, data=mtcars)
\end{verbatim}

which is meant to be read as "mpg is described using a model that
is additive in cyl, disp, and so forth.� The output is as follows:

\begin{verbatim}
Coefficients:
            Estimate Std. Error t value Pr(>|t|)
(Intercept) 12.30337   18.71788   0.657   0.5181
cyl         -0.11144    1.04502  -0.107   0.9161
disp         0.01334    0.01786   0.747   0.4635
hp          -0.02148    0.02177  -0.987   0.3350
drat         0.78711    1.63537   0.481   0.6353
wt          -3.71530    1.89441  -1.961   0.0633 .
qsec         0.82104    0.73084   1.123   0.2739
vs           0.31776    2.10451   0.151   0.8814
am           2.52023    2.05665   1.225   0.2340
gear         0.65541    1.49326   0.439   0.6652
carb        -0.19942    0.82875  -0.241   0.8122
---
Signif. codes:  0 �***� 0.001 �**� 0.01 �*� 0.05 �.� 0.1 � � 1

Residual standard error: 2.65 on 21 degrees of freedom
Multiple R-squared: 0.869,      Adjusted R-squared: 0.8066
F-statistic: 13.93 on 10 and 21 DF,  p-value: 3.793e-07
\end{verbatim}
Notice that none of the predictor variables is significant. The only one that comes even close is ``wt".

\small
\begin{verbatim}
       mpg   cyl  disp    hp  drat    wt  qsec    vs    am  gear  carb
mpg   1.00 -0.85 -0.85 -0.78  0.68 -0.87  0.42  0.66  0.60  0.48 -0.55
cyl  -0.85  1.00  0.90  0.83 -0.70  0.78 -0.59 -0.81 -0.52 -0.49  0.53
disp -0.85  0.90  1.00  0.79 -0.71  0.89 -0.43 -0.71 -0.59 -0.56  0.39
hp   -0.78  0.83  0.79  1.00 -0.45  0.66 -0.71 -0.72 -0.24 -0.13  0.75
drat  0.68 -0.70 -0.71 -0.45  1.00 -0.71  0.09  0.44  0.71  0.70 -0.09
wt   -0.87  0.78  0.89  0.66 -0.71  1.00 -0.17 -0.55 -0.69 -0.58  0.43
qsec  0.42 -0.59 -0.43 -0.71  0.09 -0.17  1.00  0.74 -0.23 -0.21 -0.66
vs    0.66 -0.81 -0.71 -0.72  0.44 -0.55  0.74  1.00  0.17  0.21 -0.57
am    0.60 -0.52 -0.59 -0.24  0.71 -0.69 -0.23  0.17  1.00  0.79  0.06
gear  0.48 -0.49 -0.56 -0.13  0.70 -0.58 -0.21  0.21  0.79  1.00  0.27
carb -0.55  0.53  0.39  0.75 -0.09  0.43 -0.66 -0.57  0.06  0.27  1.00
\end{verbatim}
\normalsize

In many cases there is a high degree of correlation between two predictor variables. The variable ``disp" has correlation coefficients of $-0.85$, $0.79$ and $0.89$ and with ``cyl", ``hp" and ``wt" respectively.

\subsection{Multicollinearity}


\subsection{Variable Selection Procedures}

There are three types of variable selection procedure.
\begin{itemize}
\item Forward Selection
\item Backward Elimination
\item Stepwise selection
\end{itemize}


The \texttt{R} command we use to perform variable selection procedures is \texttt{step()}

direction  - the mode of stepwise search, can be one of ``both", ``backward", or ``forward", with a default of ``both". If the scope argument is missing the default for direction is "backward".

\subsection{Coefficient of Determination}


\begin{verbatim}
Analysis of Variance Table

Response: mpg
          Df Sum Sq Mean Sq  F value    Pr(>F)
cyl        1 817.71  817.71 116.4245 5.034e-10 ***
disp       1  37.59   37.59   5.3526  0.030911 *
hp         1   9.37    9.37   1.3342  0.261031
drat       1  16.47   16.47   2.3446  0.140644
wt         1  77.48   77.48  11.0309  0.003244 **
qsec       1   3.95    3.95   0.5623  0.461656
vs         1   0.13    0.13   0.0185  0.893173
am         1  14.47   14.47   2.0608  0.165858
gear       1   0.97    0.97   0.1384  0.713653
carb       1   0.41    0.41   0.0579  0.812179
Residuals 21 147.49    7.02
---
Signif. codes:  0 �***� 0.001 �**� 0.01 �*� 0.05 �.� 0.1 � � 1
\end{verbatim}


\newpage
\subsection{Backward Elimination}
Our initial model includes all the predictor variables.
\begin{verbatim}
> step(fit.all)
Start:  AIC=70.9
mpg ~ cyl + disp + hp + drat + wt + qsec + vs + am + gear + carb

       Df Sum of Sq    RSS    AIC
- cyl   1    0.0799 147.57 68.915
- vs    1    0.1601 147.66 68.932
- carb  1    0.4067 147.90 68.986
- gear  1    1.3531 148.85 69.190
- drat  1    1.6270 149.12 69.249
- disp  1    3.9167 151.41 69.736
- hp    1    6.8399 154.33 70.348
- qsec  1    8.8641 156.36 70.765
<none>              147.49 70.898
- am    1   10.5467 158.04 71.108
- wt    1   27.0144 174.51 74.280
\end{verbatim}

This tables tells us the effect or removing each predictor variable individually, in terms of the AIC.
Consider the first row. This tells us the AIC value of a model fitted without the ``cyl" variable would be $68.915$.
Included in the table is effect of not removing any variables. If the ``wt" variable was to be removed, the AIC value would increase to $74.280$.
\begin{verbatim}
..
- cyl   1    0.0799 147.57 68.915
..
<none>              147.49 70.898
..
- wt    1   27.0144 174.51 74.280
\end{verbatim}

The procedure removes variables as appropriate, until it found that removing anymore variables would increase the AIC.
\begin{verbatim}
Step:  AIC=61.31
mpg ~ wt + qsec + am

       Df Sum of Sq    RSS    AIC
<none>              169.29 61.307
- am    1    26.178 195.46 63.908
- qsec  1   109.034 278.32 75.217
- wt    1   183.347 352.63 82.790
\end{verbatim}

The outcome of this procedure is that ``mpg" is best explained as a linear combination of the ``am", ``qsec" and ``wt" variables.

\begin{verbatim}
Coefficients:
(Intercept)           wt         qsec           am
      9.618       -3.917        1.226        2.936  
\end{verbatim}

%\[
%\hat{mpg} = 9.618 - 3.917wt + 1.226qsec + 2.936am
%\]

\end{document} 